% A tikzlibrary[libraryName].code.tex is loaded automatically by tikz when using
% \usetikzlibrary{libraryName}. Therefore, you should just use
% \usetikzlibrary{zx} to load this library. We also provide a package to
% directly load this library using \usepackage{zx}.

\RequirePackage{amssymb} % For short minus
\RequirePackage{etoolbox}
\RequirePackage{xparse} % For NewDocumentComments
\RequirePackage{bm} % For bold math fonts

\usetikzlibrary{cd,backgrounds,positioning,shapes}
% Declare layers.
\pgfdeclarelayer{background}
\pgfdeclarelayer{main}

%%%%%%%%%%%%%%%%%%%%%%%%%%%%%%
%%%% User modifiable variables
%%%%%%%%%%%%%%%%%%%%%%%%%%%%%%
% Define colors, can be redefine by user
\definecolor{colorZxZ}{RGB}{204,255,204}
\definecolor{colorZxX}{RGB}{255,136,136}
\definecolor{colorZxH}{RGB}{255,255,0}

%%% Some wires (the one having an intermediate H, X, or S gate) may need some additional space for
%%% specific columns.
%%% Use these spaces like &[\zxHCol] or \\[\zxHRow] in that case
%% Defines the space to add for columns and rows containing a connection with Hadamard
% This is for "curved" wires
\newcommand{\zxHCol}{1mm}
\newcommand{\zxHRow}{1mm}
% This is for "flat" wires (usually takes more space)
\newcommand{\zxHColFlat}{1.5mm}
\newcommand{\zxHRowFlat}{1.5mm}
%% Defines the space to add for columns and rows containing a connection with small X/Z
\newcommand{\zxSCol}{1mm}
\newcommand{\zxSRow}{1mm}
\newcommand{\zxSColFlat}{1.5mm}
\newcommand{\zxSRowFlat}{1.5mm}
%% Defines the space to add for columns having both H and Spiders
\newcommand{\zxHSCol}{1mm}
\newcommand{\zxHSRow}{1mm}
\newcommand{\zxHSColFlat}{1mm}
\newcommand{\zxHSRowFlat}{1mm}
%% Wires only: when adding only wires with empty nodes, the space between columns can be too small. Useful not to shrink swap gates...
\newcommand{\zxWCol}{.55em}
\newcommand{\zxWRow}{.55em}
%% When vdots/dots are used in lines
\newcommand{\zxDotsCol}{3mm}
\newcommand{\zxDotsRow}{3mm}


% Angles by default for s and o related arrows
\def\zxDefaultSoftAngleS{30}
\def\zxDefaultSoftAngleO{40}

% Scale to use when scaling 3 dots
\def\zxScaleDots{.7}

% 0.4pt is default in tikz. Also used to ensure it has not been modified document wise by other libraries
% (quantikz notably changes this parameter).
\newcommand{\zxDefaultLineWidth}{0.4pt}

%%%%%%%%%%%%%%%%%%%%%%%%%%%%%%
%%%% Tikz styles
%%%%%%%%%%%%%%%%%%%%%%%%%%%%%%

% Styles. User should not modify "wires definition", but is free to change:
% - "/zx/default style nodes/" to change completely the node style
% - "/zx/user overlay nodes" to add stuff on current node style
% - "/zx/default style wires" to change the wire style
% - "/zx/user overlay wires/" to add stuff on wire style
% The user is not supposed to use node styles directly (use \zxZ{}, \zxZ{\alpha+\beta}, \zxFrac-{\pi}{4}...)
% but is free (and encouraged) to use the styles in "wires definition" like \ar[r,o'].
\tikzset{
  /zx/wires definition/.style={
    %%% Basic default properties
    draw,
    -,
    line width=\zxDefaultLineWidth,
    %%% Useful shortcut (shorter lines means easy "align" of & symbols. Love M-x align in emacs btw.)
    l/.style={looseness=##1},
    looseness wires only/.style={% Looseness used for wires only.
      looseness=1.2,
    },
    lw/.style={looseness wires only},
    % Use this when you are drawing lines between none nodes only (like swap gates...)
    between none/.style={
      looseness wires only,
      wire centered
    },
    bn/.style={
      between none
    },
    % ------------------------------
    % Practical stuff to draw lines easily:
    % Prefer to use these are they can be easily customized for each style and shorter to type.
    % Note that the letter is supposed to represent the shape of the link
    % dots/dashes are used to specify the position of the arrow.
    % Typically ' means top, . bottom, X- is right to X (or should arrive with angle 0),
    % -X is left to X (or should leave with angle zero). These shapes are usually designed to
    % work when the starting node is left most (or above of both nodes have the same column).
    % But they may work both way for some of them.
    % ------------------------------
    %%% Cup/Cap
    % Like a C shape. Useful for Bell states whose angles must be really marked.
    C/.style={/tikz/in=180,/tikz/out=180,looseness=2},
    % Like C, but symetric
    C-/.style={/tikz/in=0,/tikz/out=0,looseness=2},
    C'/.style={/tikz/in=90,/tikz/out=90,looseness=2},
    C./.style={/tikz/in=-90,/tikz/out=-90,looseness=2},
    % Similar to C, but with a softer angle. The '.- marker represents the portion of
    % the circle (hence the o) to keep (top, bottom,left/right).
    % Angle is customizable, for instance o'=50.
    o'/.style={/tikz/out=##1,/tikz/in=180-##1},
    o'/.default=\zxDefaultSoftAngleO,
    o./.style={/tikz/out=-##1,/tikz/in=180+##1},
    o./.default=\zxDefaultSoftAngleO,
    -o/.style={/tikz/out=-90-##1,/tikz/in=90+##1},
    -o/.default=\zxDefaultSoftAngleO,
    o-/.style={/tikz/out=-90+##1,/tikz/in=90-##1},
    o-/.default=\zxDefaultSoftAngleO,
    % Similar to o, but can be used also for diagonal items.
    % Why ()? Visualize fixing the top part and moving the bottom part.
    (/.style={bend right},
    )/.style={bend left},
    % Links with a s-like shape.
    s/.style={/tikz/out=0,/tikz/in=180,looseness=0.6},
    % -s'.- shapes are like s, but with a soften (customizable like o) angle.
    % The '. say if you are going up or down, and - forces a sharp angle (- is flat) on the side of the -
    s'/.style={/tikz/out=##1,/tikz/in=180+##1},
    s'/.default=\zxDefaultSoftAngleS,
    s./.style={/tikz/out=-##1,/tikz/in=180-##1},
    s./.default=\zxDefaultSoftAngleS,
    -s'/.style={/tikz/out=0,/tikz/in=180+##1},
    -s'/.default=\zxDefaultSoftAngleS,
    -s./.style={/tikz/out=0,/tikz/in=180-##1},
    -s./.default=\zxDefaultSoftAngleS,
    s'-/.style={/tikz/out=##1,/tikz/in=180},
    s'-/.default=\zxDefaultSoftAngleS,
    s.-/.style={/tikz/out=-##1,/tikz/in=180},
    s.-/.default=\zxDefaultSoftAngleS,
    % Links with a s-like shape... but read from top to bottom
    ss/.style={/tikz/out=0-90,/tikz/in=180-90,looseness=0.6},
    % -s'.- shapes are like s, but with a soften (customizable like o) angle.
    % The '. say if you are going up or down, and - forces a sharp angle (- is flat) on the side of the -
    ss./.style={/tikz/out=##1-90,/tikz/in=180-90+##1},
    ss./.default=\zxDefaultSoftAngleS,
    .ss/.style={/tikz/out=-##1-90,/tikz/in=180-90-##1},
    .ss/.default=\zxDefaultSoftAngleS,
    sIs./.style={/tikz/out=0-90,/tikz/in=180-90+##1},
    sIs./.default=\zxDefaultSoftAngleS,
    .sIs/.style={/tikz/out=0-90,/tikz/in=180-90-##1},
    .sIs/.default=\zxDefaultSoftAngleS,
    ss.I/.style={/tikz/out=##1-90,/tikz/in=180-90},
    ss.I/.default=\zxDefaultSoftAngleS,
    I.ss/.style={/tikz/out=-##1-90,/tikz/in=180-90},
    I.ss/.default=\zxDefaultSoftAngleS,
    % No line but vdots/dots in between.
    3 vdots/.style={draw=none, "\makebox[0pt][r]{##1}\scalebox{\zxScaleDots}{$\cvdots$}" anchor=center},
    3 vdots/.default={},
    3 dots/.style={draw=none, "\makebox[0pt][r]{##1}\scalebox{\zxScaleDots}{$\chdots$}" anchor=center},
    3 dots/.default={},
    % Add a Hadmard/Z/X (no phase) in the middle of the line. Practical to add small nodes without creating
    % a new column/row. However, make sure the corresponding row/column is larger, using &[\zxHCol]
    % for columns and \\[\zxHRow] for rows (for Z/X style, use zxSCol and sxSRow), if you have both spiders
    % and Hadamard, use \zxHSCol and \zxHSRow.
    H/.style={"" {zxHSmall,anchor=center}},
    Z/.style={"" {zxNoPhaseSmallZ,anchor=center}},
    X/.style={"" {zxNoPhaseSmallX,anchor=center}},
    % Arrow will go out from the center of the shape instead of from the border. Useful
    % when connecting nodes with different shapes, it will give back a symetric connection.
    wire centered/.style={
      on layer=background,
      /tikz/commutative diagrams/start anchor=center,
      /tikz/commutative diagrams/end anchor=center,
    },
    wire centered start/.style={
      on layer=background,
      /tikz/commutative diagrams/start anchor=center,
    },
    wire centered end/.style={
      on layer=background,
      /tikz/commutative diagrams/end anchor=center,
    },
    wc/.style={wire centered},
    wcs/.style={wire centered start},
    wce/.style={wire centered end},
    wire not centered/.style={
      /tikz/commutative diagrams/start anchor=,
      /tikz/commutative diagrams/end anchor=,
    },
  },
  /zx/styles/rounded style/.style={
    %% Can be redefined by user
    % Style for empty nodes
    zxAllNodes/.style={
      shape=rectangle, % Otherwise nodes are asymetrical rectangle, which is not practical in our case. Gives notably anchor "center" which is really centered compared to asymatrical rectangles
      anchor=center,   % Center cells
      line width=\zxDefaultLineWidth,
      execute at begin node={\thinmuskip=0mu\medmuskip=0mu\thickmuskip=0mu}, % Reduce space around +/-...
    },
    % Use this to denote an empty diagram
    zxEmptyDiagram/.style={
      zxAllNodes,
      draw,
      dashed,
      minimum size=4mm,
    },
    % Style to use when no node is drawn
    zxNone/.style={
      zxAllNodes,
      inner sep=0mm,
      outer sep=0mm
    },
    % Style to use when no node is drawn, but a bit of space is required not to make the diagram too small
    zxNone+/.style={
      zxAllNodes,
      inner sep=1mm,
      outer sep=0mm
    },
    % Like zxNone+, but without width (wold prefer |, but special car in |[]|...
    zxNoneI/.style={
      zxNone+,
      inner xsep=0mm,
    },
    % Like zxNone+, but without height
    zxNone-/.style={
      zxNone+,
      inner ysep=0mm,
    },
    % Style to use when no node is drawn, but a large space must be reserved (typically used to fake two
    % nodes on a single line)
    zxNoneDouble/.style={
      zxAllNodes,
      inner sep=0mm,
      outer sep=0mm
    },
    % Style to use when no node is drawn, but a bit of space is required not to make the diagram too small
    zxNoneDouble+/.style={
      zxAllNodes,
      inner sep=.6em,
      outer sep=0mm
    },
    % Like zxNoneDouble+, but without width (wold prefer |, but special car in |[]|...
    zxNoneDoubleI/.style={
      zxNoneDouble+,
      inner xsep=0mm,
    },
    % Like zxNoneDouble+, but without height
    zxNoneDouble-/.style={
      zxNoneDouble+,
      inner ysep=0mm,
    },
    % Will be specific to all spiders
    zxSpiders/.style={
      draw=black,
    },
    % Will use this style when drawing a X/Z node without phase (not for end user directly)
    zxNoPhase/.style={
      zxAllNodes,
      zxSpiders,
      inner sep=0mm,
      minimum size=2mm,
      shape=circle,
    },
    % Used only in decoration of wires, to add small empty X/Z nodes.
    zxNoPhaseSmall/.style={
      zxNoPhase
    },
    % Style for nodes that are small enough to fit in a circle, like $\zxMinus \frac{\pi}{4}$
    % zxShort/.style={anchor=center,minimum size=5mm, font={\footnotesize\boldmath}, shape=rectangle, rounded corners=2mm, inner sep=0.2mm, outer sep=-2mm, scale=0.8, draw=black},
    zxShort/.style={
      zxAllNodes,
      zxSpiders,
      minimum size=5mm,
      font={\footnotesize\boldmath},
      rounded rectangle,
      inner sep=0.0mm,
      scale=0.8,
    }, % negative outer sep would draw lines from inside...
    % Style for nodes that are bigger, like $\alpha+\beta$ or $(a\oplus b)\pi$
    zxLong/.style={zxShort, inner xsep=1.2mm},
    %%% Style that was supposed to chooses which style to apply depending on the input text
    %%% Can't find how to do.
    % zxChoose/.code={
    %   %% To fix
    % },
    %%%%%%%%%%% Style defined depending on above ones. Feel free to redefine them yourself.
    zxNoPhaseZ/.style={zxNoPhase,fill=colorZxZ},
    zxNoPhaseX/.style={zxNoPhase,fill=colorZxX},
    zxNoPhaseSmallZ/.style={zxNoPhaseSmall,fill=colorZxZ},
    zxNoPhaseSmallX/.style={zxNoPhaseSmall,fill=colorZxX},
    zxShortZ/.style={zxShort,fill=colorZxZ},
    zxShortX/.style={zxShort,fill=colorZxX},
    zxLongZ/.style={zxLong,fill=colorZxZ},
    zxLongX/.style={zxLong,fill=colorZxX},
    %%% Was supposed to automatically find the good style depending on content... Can't find how to do.
    % Styles zxLong{X/Z} zxNoPhase{X/Z} are automatically selected by \zxZ{...} and \zxX{...} commands
    % and zxShort is selected for fractions only like in \zxFracZ-{\pi}{4}
    % zxZ/.style={zxChoose={##1},fill=colorZxZ},
    % zxX/.style={zxChoose={##1},fill=colorZxX},
    % For Hadamard
    zxH/.style={
      zxAllNodes,
      outer sep=0pt,
      fill=colorZxH,
      draw,
      inner sep=0.6mm,
      minimum height=1.5mm,
      minimum width=1.5mm,
      shape=rectangle},
    zxHSmall/.style={zxH},
  },
  % Default style. Can be changed by user
  /zx/default style nodes/.style={
    /zx/styles/rounded style
  },
  % User can put here any additional property
  /zx/user overlay nodes/.style={
  },
  % Default wire style. Can be changed by user.
  /zx/default style wires/.style={
  },
  % User can add stuff in this style to improve wire styles
  /zx/user overlay wires/.style={
  },
  /zx/defaultEnv/.style={
    column sep=tiny,
    row sep=tiny,
    % center on the math axis
    baseline={([yshift=-axis_height]current bounding box.center)},
    % Fix 1-row diagram baseline
    1-row diagram/.style={%
      /tikz/baseline={([yshift=-axis_height]current bounding box.center)}%
    },
    % Load (thanks ".search also") our own style
    /tikz/every node/.style={%
      /zx/default style nodes,
      /zx/user overlay nodes,
    },
    every arrow/.style={%
      /zx/wires definition,
      /zx/default style wires,
      /zx/user overlay wires,
    },
  },
}

%%%%%%%%%%%%%%%%%%%%%%%%%%%%%%
%%%% Helper functions
%%%%%%%%%%%%%%%%%%%%%%%%%%%%%%

% Defines a "on layer=nameoflayer" style. TODO: check if better to move it in /zx/
% https://tex.stackexchange.com/questions/20425/z-level-in-tikz/20426#20426
\pgfkeys{%
  /tikz/on layer/.code={
    \def\tikz@path@do@at@end{\endpgfonlayer\endgroup\tikz@path@do@at@end}%
    \pgfonlayer{#1}\begingroup%
  }%
}

%%% Declare a symbol for a short minus (useful in fractions)
\DeclareMathSymbol{\zxMinus}{\mathbin}{AMSa}{"39} % Requires amssymb

%%% Create different kinds of dots...
%% https://tex.stackexchange.com/questions/617959
%% https://tex.stackexchange.com/questions/528774/excess-vertical-space-in-vdots/528775#528775
\DeclareRobustCommand\cvdotsAboveBaseline{%
  \vbox{\baselineskip4\p@ \lineskiplimit\z@%
    \hbox{.}\hbox{.}\hbox{.}}
}

\DeclareRobustCommand{\cvdotsCenterMathline}{%
  % vcenter is used to center the argument on the 'math axis', which is at half the height of an 'x', or about the position of a minus sign.
  \vcenter{\cvdotsAboveBaseline}%
}

\DeclareRobustCommand{\cvdotsCenterBaseline}{%
  \raisebox{-.5\height}{%
    $\cvdotsAboveBaseline$%
  }%
}

\DeclareRobustCommand{\chdots}{%
  \raisebox{-.5\height}{%
    \rotatebox{90}{% Maybe better options than rotatebox...
      $\cvdotsAboveBaseline$%
    }%
  }%
}

\DeclareRobustCommand{\cvdots}{\cvdotsCenterMathline}


%%%%%%%%%%%%%%%%%%%%%%%%%%%%%%%%%%%%%%%%%%%%%%%%%%%%%%%%%%%%%%%%%%%%%%%%%
%%% Practical macros to automatically choose appropriate style and arrows
%%%%%%%%%%%%%%%%%%%%%%%%%%%%%%%%%%%%%%%%%%%%%%%%%%%%%%%%%%%%%%%%%%%%%%%%%
% /!\ Warning: you should add {} at the end of all macros (except arrows)!
% Not using that may work for now, but it may break later...
% TODO: define them only in \zx environment.

% % Example: \leftManyDots{n}
% Useful to put on the left of a node like "n \vdots", linked to the next node.  Example: \leftManyDots{n}.
% First optional argument is scale of text, second is scale of =.
\NewExpandableDocumentCommand{\leftManyDots}{O{1}O{\zxScaleDots}m}{%
  |[zxNone+,inner xsep=0pt]| \scalebox{#1}{$#3$\,}\makebox[0pt][l]{\scalebox{#2}{$\cvdots$}} \ar[r,-s.,start anchor=north east] \ar[r,-s',start anchor=south east] \pgfmatrixnextcell%
}

% Useful to link two nodes and put a vdots in between.
\NewExpandableDocumentCommand{\middleManyDots}{}{%
  \ar[r,3 vdots] \ar[o',r] \ar[o.,r]%
}

% Like \leftManyDots but on the right. Do *not* create a new node, like in |[zxShortZ]| \alpha \rightManyDots{m}
\NewExpandableDocumentCommand{\rightManyDots}{O{1}O{\zxScaleDots}m}{%
  \ar[r,s'-,end anchor=north west] \ar[r,s.-,end anchor=south west] \pgfmatrixnextcell |[zxNone+,inner xsep=0pt]| \makebox[0pt][r]{\scalebox{#2}{$\cvdots$}}\scalebox{#1}{\,$#3$}
}

% A swap on one line... Practical mostly to gain space. Must be used with large nodes tough...
% \NewExpandableDocumentCommand{\OneLineSwap}{}{%
%   \ar[r,s,start anchor=south,end anchor=north] \ar[r,s,start anchor=north,end anchor=south]
% }


\NewExpandableDocumentCommand{\zxLoop}{O{90}O{20}O{}m}{%
  \ar[loop,in=#1-#2,out=#1+#2,looseness=8,min distance=3mm,#3]
}

\NewExpandableDocumentCommand{\zxLoopAboveDots}{O{20}O{}m}{%
  \ar[loop,in=90-#1,out=90+#1,looseness=8,min distance=3mm,"\cvdots" {scale=.6,anchor=north,yshift=-0.4mm},#2]
}

% Usage: node without any style, but may have space. Default is no space, \zxNone+{} is both horizontal
% and vertical, \zxNone-{} is only horizontal space, \zxNone|{} is only vertical space.
\NewExpandableDocumentCommand{\zxNone}{t+t-t|O{}m}{
  \IfBooleanTF{#1}{ % \zxNone+
    |[zxNone+,#4]| #5%
  }{
    \IfBooleanTF{#2}{ % \zxNone-
      |[zxNone-,#4]| #5%
    }{
      \IfBooleanTF{#3}{ % \zxNone
        |[zxNoneI,#4]| #5%
      }{% \zxNone
        |[zxNone,#4]| #5%
      }
    }
  }
}

% Cf \zxNone, but with larger space.
\NewExpandableDocumentCommand{\zxNoneDouble}{t+t-t|O{}m}{
  \IfBooleanTF{#1}{ % \zxNoneDouble+
    |[zxNoneDouble+,#4]| #5%
  }{
    \IfBooleanTF{#2}{ % \zxNoneDouble-
      |[zxNoneDouble-,#4]| #5%
    }{
      \IfBooleanTF{#3}{ % \zxNoneDouble
        |[zxNoneDoubleI,#4]| #5%
      }{% \zxNoneDouble
        |[zxNoneDouble,#4]| #5%
      }
    }
  }
}


\NewExpandableDocumentCommand{\zxX}{O{}m}{
  \ifblank{#2}{
    |[zxNoPhaseX,#1]| {}%
  }{%
    |[zxLongX,#1]| #2%
  }%
}

\NewExpandableDocumentCommand{\zxZ}{O{}m}{
  \ifblank{#2}{
    |[zxNoPhaseZ,#1]| {}%
  }{%
    |[zxLongZ,#1]| #2%
  }%
}

\NewExpandableDocumentCommand{\zxH}{O{}m}{
  |[zxH,#1]| {}%
}

% Use like: \zxFracX{\pi}{4} for positive values or for negative \zxFracX-{\pi}{4}
\NewExpandableDocumentCommand{\zxFracX}{t-mm}{
  |[zxShortX]| \IfBooleanTF{#1}{\zxMinus}{}\frac{#2}{#3}%
}

% Use like: \zxFracZ{\pi}{4} for positive values or for negative \zxFracZ-{\pi}{4}
\NewExpandableDocumentCommand{\zxFracZ}{t-mm}{
  |[zxShortZ]| \IfBooleanTF{#1}{\zxMinus}{}\frac{#2}{#3}%
}

\NewExpandableDocumentCommand{\zxEmptyDiagram}{}{
  |[zxEmptyDiagram]| {}%
}

% Quantikz has a bug which adds space automatically.
% https://tex.stackexchange.com/questions/618330
% Fixing that by copying the original (unpatched) functions, and reusing them later.
% Warning: you must load this package **before** quantikz otherwise the fix will not work.
\let\tikzcd@@originalCopyZx\tikzcd@
\let\endtikzcd@originalCopyZx\endtikzcd

%%%%% Main environment \begin{ZX}...\end{ZX}
\NewDocumentEnvironment{ZX}{O{}}{%
  \bgroup%
  % Add a switch in case someone really wants the current tikzcd version:
  \ifdefined\doNotPatchQuantikz% Do not patch tikzcd.
  \else% Restore locally original tikzcd.
    \let\tikzcd@\tikzcd@@originalCopyZx%
    \let\endtikzcd\endtikzcd@originalCopyZx%
  \fi%
  \pgfsetlayers{background,main} % Layers are defined locally to avoid to disturb other drawings
  \begin{tikzcd}[%
    /zx/defaultEnv,%
    #1]%
  }{\end{tikzcd}\egroup}

%%%%% Shortcut macro \zx{...} equivalent to \begin{ZX}...\end{ZX}
\newcommand\zx{%
  \begingroup% To avoid ampersand issues https://tex.stackexchange.com/a/611535/116348
  \NewDocumentCommand{\tmpZX}{O{}+m}{%
    \endgroup%
    \begin{ZX}[##1]%
      ##2%
    \end{ZX}%
  }%
  \catcode`&=13%
  \tmpZX%
}

%%%%%%%%%%%%%%%%%%%%%%%%%%%%%%
%%% Old code that tried to automatically find if zxShort or zxLong should be used...
%%% Now using a special command for fractions (easier to code, and more customizable)
%%%%%%%%%%%%%%%%%%%%%%%%%%%%%%
\newsavebox\zx@box % Temporary box to compute height/width/depth

\newlength{\zxMaxDepthPlusHeight}\setlength{\zxMaxDepthPlusHeight}{2em}
\def\zxMaxRatio{1.3} % Ratio width/(height+depth)

\NewExpandableDocumentCommand{\zxChooseStyle}{mmmm}{%
  % #1=text,#2=empty style,#3=short style,#4=long style
  \savebox\zx@box{#1}%
  % Check if width is 0pt:
  \ifdimcomp{\wd\zx@box}{=}{0pt}{% Return empty style if box is empty
    #2%
  }{% Else compute size of thext
    % Check if height+depth < zxMaxDepthPlusHeight to see if short style applies
    \ifdimcomp{\dimexpr\dp\zx@box+\ht\zx@box\relax}{<}{\zxMaxDepthPlusHeight}{%
      % Check if width < ratio*(height+depth) to see if short style applies
      \ifdimcomp{\wd\zx@box}{<}{\dimexpr \zxMaxRatio\ht\zx@box + \zxMaxRatio\dp\zx@box\relax}{%
        #3% Short style is used
      }{ % Else
        #4% Long style is used
      } %
    }{
      #4% Long style is used
    }
  }%
}
