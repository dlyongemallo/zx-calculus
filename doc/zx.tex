\documentclass[a4paper]{ltxdoc}

%%%%%%%%%%%%%%%%%%%%%%%%%%%%%%
%%% Packages
%%%%%%%%%%%%%%%%%%%%%%%%%%%%%%

\usepackage[margin=3cm]{geometry}
\usepackage{calc}
\usepackage{tikz}
\usetikzlibrary{shadows,fit}
% \usetikzlibrary fails because file is not in current directory, lazy to setup TEXINPUTS
\makeatletter
  % A tikzlibrary[libraryName].code.tex is loaded automatically by tikz when using
% \usetikzlibrary{libraryName}. Therefore, you should just use
% \usetikzlibrary{zx} to load this library. We also provide a package to
% directly load this library using \usepackage{zx}.

\RequirePackage{amssymb} % For short minus
\RequirePackage{etoolbox}
\RequirePackage{xparse} % For NewDocumentComments
\RequirePackage{bm} % For bold math fonts

\usetikzlibrary{cd,backgrounds,positioning,shapes}
% Declare layers.
\pgfdeclarelayer{background}
\pgfdeclarelayer{main}

%%%%%%%%%%%%%%%%%%%%%%%%%%%%%%
%%%% User modifiable variables
%%%%%%%%%%%%%%%%%%%%%%%%%%%%%%
% Define colors, can be redefine by user
\definecolor{colorZxZ}{RGB}{204,255,204}
\definecolor{colorZxX}{RGB}{255,136,136}
\definecolor{colorZxH}{RGB}{255,255,0}

%%% Some wires (the one having an intermediate H, X, or S gate) may need some additional space for
%%% specific columns.
%%% Use these spaces like &[\zxHCol] or \\[\zxHRow] in that case
%% Defines the space to add for columns and rows containing a connection with Hadamard
% This is for "curved" wires
\newcommand{\zxHCol}{1mm}
\newcommand{\zxHRow}{1mm}
% This is for "flat" wires (usually takes more space)
\newcommand{\zxHColFlat}{1.5mm}
\newcommand{\zxHRowFlat}{1.5mm}
%% Defines the space to add for columns and rows containing a connection with small X/Z
\newcommand{\zxSCol}{1mm}
\newcommand{\zxSRow}{1mm}
\newcommand{\zxSColFlat}{1.5mm}
\newcommand{\zxSRowFlat}{1.5mm}
%% Defines the space to add for columns having both H and Spiders
\newcommand{\zxHSCol}{1mm}
\newcommand{\zxHSRow}{1mm}
\newcommand{\zxHSColFlat}{1mm}
\newcommand{\zxHSRowFlat}{1mm}
%% Wires only: when adding only wires with empty nodes, the space between columns can be too small. Useful not to shrink swap gates...
\newcommand{\zxWCol}{.55em}
\newcommand{\zxWRow}{.55em}
%% When vdots/dots are used in lines
\newcommand{\zxDotsCol}{3mm}
\newcommand{\zxDotsRow}{3mm}


% Angles by default for s and o related arrows
\def\zxDefaultSoftAngleS{30}
\def\zxDefaultSoftAngleO{40}

% Scale to use when scaling 3 dots
\def\zxScaleDots{.7}

% 0.4pt is default in tikz. Also used to ensure it has not been modified document wise by other libraries
% (quantikz notably changes this parameter).
\newcommand{\zxDefaultLineWidth}{0.4pt}

%%%%%%%%%%%%%%%%%%%%%%%%%%%%%%
%%%% Tikz styles
%%%%%%%%%%%%%%%%%%%%%%%%%%%%%%

% Styles. User should not modify "wires definition", but is free to change:
% - "/zx/default style nodes/" to change completely the node style
% - "/zx/user overlay nodes" to add stuff on current node style
% - "/zx/default style wires" to change the wire style
% - "/zx/user overlay wires/" to add stuff on wire style
% The user is not supposed to use node styles directly (use \zxZ{}, \zxZ{\alpha+\beta}, \zxFrac-{\pi}{4}...)
% but is free (and encouraged) to use the styles in "wires definition" like \ar[r,o'].
\tikzset{
  /zx/wires definition/.style={
    %%% Basic default properties
    draw,
    -,
    line width=\zxDefaultLineWidth,
    %%% Useful shortcut (shorter lines means easy "align" of & symbols. Love M-x align in emacs btw.)
    l/.style={looseness=##1},
    looseness wires only/.style={% Looseness used for wires only.
      looseness=1.2,
    },
    lw/.style={looseness wires only},
    % Use this when you are drawing lines between none nodes only (like swap gates...)
    between none/.style={
      looseness wires only,
      wire centered
    },
    bn/.style={
      between none
    },
    % ------------------------------
    % Practical stuff to draw lines easily:
    % Prefer to use these are they can be easily customized for each style and shorter to type.
    % Note that the letter is supposed to represent the shape of the link
    % dots/dashes are used to specify the position of the arrow.
    % Typically ' means top, . bottom, X- is right to X (or should arrive with angle 0),
    % -X is left to X (or should leave with angle zero). These shapes are usually designed to
    % work when the starting node is left most (or above of both nodes have the same column).
    % But they may work both way for some of them.
    % ------------------------------
    %%% Cup/Cap
    % Like a C shape. Useful for Bell states whose angles must be really marked.
    C/.style={/tikz/in=180,/tikz/out=180,looseness=2},
    % Like C, but symetric
    C-/.style={/tikz/in=0,/tikz/out=0,looseness=2},
    C'/.style={/tikz/in=90,/tikz/out=90,looseness=2},
    C./.style={/tikz/in=-90,/tikz/out=-90,looseness=2},
    % Similar to C, but with a softer angle. The '.- marker represents the portion of
    % the circle (hence the o) to keep (top, bottom,left/right).
    % Angle is customizable, for instance o'=50.
    o'/.style={/tikz/out=##1,/tikz/in=180-##1},
    o'/.default=\zxDefaultSoftAngleO,
    o./.style={/tikz/out=-##1,/tikz/in=180+##1},
    o./.default=\zxDefaultSoftAngleO,
    -o/.style={/tikz/out=-90-##1,/tikz/in=90+##1},
    -o/.default=\zxDefaultSoftAngleO,
    o-/.style={/tikz/out=-90+##1,/tikz/in=90-##1},
    o-/.default=\zxDefaultSoftAngleO,
    % Similar to o, but can be used also for diagonal items.
    % Why ()? Visualize fixing the top part and moving the bottom part.
    (/.style={bend right},
    )/.style={bend left},
    % Links with a s-like shape.
    s/.style={/tikz/out=0,/tikz/in=180,looseness=0.6},
    % -s'.- shapes are like s, but with a soften (customizable like o) angle.
    % The '. say if you are going up or down, and - forces a sharp angle (- is flat) on the side of the -
    s'/.style={/tikz/out=##1,/tikz/in=180+##1},
    s'/.default=\zxDefaultSoftAngleS,
    s./.style={/tikz/out=-##1,/tikz/in=180-##1},
    s./.default=\zxDefaultSoftAngleS,
    -s'/.style={/tikz/out=0,/tikz/in=180+##1},
    -s'/.default=\zxDefaultSoftAngleS,
    -s./.style={/tikz/out=0,/tikz/in=180-##1},
    -s./.default=\zxDefaultSoftAngleS,
    s'-/.style={/tikz/out=##1,/tikz/in=180},
    s'-/.default=\zxDefaultSoftAngleS,
    s.-/.style={/tikz/out=-##1,/tikz/in=180},
    s.-/.default=\zxDefaultSoftAngleS,
    % Links with a s-like shape... but read from top to bottom
    ss/.style={/tikz/out=0-90,/tikz/in=180-90,looseness=0.6},
    % -s'.- shapes are like s, but with a soften (customizable like o) angle.
    % The '. say if you are going up or down, and - forces a sharp angle (- is flat) on the side of the -
    ss./.style={/tikz/out=##1-90,/tikz/in=180-90+##1},
    ss./.default=\zxDefaultSoftAngleS,
    .ss/.style={/tikz/out=-##1-90,/tikz/in=180-90-##1},
    .ss/.default=\zxDefaultSoftAngleS,
    sIs./.style={/tikz/out=0-90,/tikz/in=180-90+##1},
    sIs./.default=\zxDefaultSoftAngleS,
    .sIs/.style={/tikz/out=0-90,/tikz/in=180-90-##1},
    .sIs/.default=\zxDefaultSoftAngleS,
    ss.I/.style={/tikz/out=##1-90,/tikz/in=180-90},
    ss.I/.default=\zxDefaultSoftAngleS,
    I.ss/.style={/tikz/out=-##1-90,/tikz/in=180-90},
    I.ss/.default=\zxDefaultSoftAngleS,
    % No line but vdots/dots in between.
    3 vdots/.style={draw=none, "\makebox[0pt][r]{##1}\scalebox{\zxScaleDots}{$\cvdots$}" anchor=center},
    3 vdots/.default={},
    3 dots/.style={draw=none, "\makebox[0pt][r]{##1}\scalebox{\zxScaleDots}{$\chdots$}" anchor=center},
    3 dots/.default={},
    % Add a Hadmard/Z/X (no phase) in the middle of the line. Practical to add small nodes without creating
    % a new column/row. However, make sure the corresponding row/column is larger, using &[\zxHCol]
    % for columns and \\[\zxHRow] for rows (for Z/X style, use zxSCol and sxSRow), if you have both spiders
    % and Hadamard, use \zxHSCol and \zxHSRow.
    H/.style={"" {zxHSmall,anchor=center}},
    Z/.style={"" {zxNoPhaseSmallZ,anchor=center}},
    X/.style={"" {zxNoPhaseSmallX,anchor=center}},
    % Arrow will go out from the center of the shape instead of from the border. Useful
    % when connecting nodes with different shapes, it will give back a symetric connection.
    wire centered/.style={
      on layer=background,
      /tikz/commutative diagrams/start anchor=center,
      /tikz/commutative diagrams/end anchor=center,
    },
    wire centered start/.style={
      on layer=background,
      /tikz/commutative diagrams/start anchor=center,
    },
    wire centered end/.style={
      on layer=background,
      /tikz/commutative diagrams/end anchor=center,
    },
    wc/.style={wire centered},
    wcs/.style={wire centered start},
    wce/.style={wire centered end},
    wire not centered/.style={
      /tikz/commutative diagrams/start anchor=,
      /tikz/commutative diagrams/end anchor=,
    },
  },
  /zx/styles/rounded style/.style={
    %% Can be redefined by user
    % Style for empty nodes
    zxAllNodes/.style={
      shape=rectangle, % Otherwise nodes are asymetrical rectangle, which is not practical in our case. Gives notably anchor "center" which is really centered compared to asymatrical rectangles
      anchor=center,   % Center cells
      line width=\zxDefaultLineWidth,
      execute at begin node={\thinmuskip=0mu\medmuskip=0mu\thickmuskip=0mu}, % Reduce space around +/-...
    },
    % Use this to denote an empty diagram
    zxEmptyDiagram/.style={
      zxAllNodes,
      draw,
      dashed,
      minimum size=4mm,
    },
    % Style to use when no node is drawn
    zxNone/.style={
      zxAllNodes,
      inner sep=0mm,
      outer sep=0mm
    },
    % Style to use when no node is drawn, but a bit of space is required not to make the diagram too small
    zxNone+/.style={
      zxAllNodes,
      inner sep=1mm,
      outer sep=0mm
    },
    % Like zxNone+, but without width (wold prefer |, but special car in |[]|...
    zxNoneI/.style={
      zxNone+,
      inner xsep=0mm,
    },
    % Like zxNone+, but without height
    zxNone-/.style={
      zxNone+,
      inner ysep=0mm,
    },
    % Style to use when no node is drawn, but a large space must be reserved (typically used to fake two
    % nodes on a single line)
    zxNoneDouble/.style={
      zxAllNodes,
      inner sep=0mm,
      outer sep=0mm
    },
    % Style to use when no node is drawn, but a bit of space is required not to make the diagram too small
    zxNoneDouble+/.style={
      zxAllNodes,
      inner sep=.6em,
      outer sep=0mm
    },
    % Like zxNoneDouble+, but without width (wold prefer |, but special car in |[]|...
    zxNoneDoubleI/.style={
      zxNoneDouble+,
      inner xsep=0mm,
    },
    % Like zxNoneDouble+, but without height
    zxNoneDouble-/.style={
      zxNoneDouble+,
      inner ysep=0mm,
    },
    % Will be specific to all spiders
    zxSpiders/.style={
      draw=black,
    },
    % Will use this style when drawing a X/Z node without phase (not for end user directly)
    zxNoPhase/.style={
      zxAllNodes,
      zxSpiders,
      inner sep=0mm,
      minimum size=2mm,
      shape=circle,
    },
    % Used only in decoration of wires, to add small empty X/Z nodes.
    zxNoPhaseSmall/.style={
      zxNoPhase
    },
    % Style for nodes that are small enough to fit in a circle, like $\zxMinus \frac{\pi}{4}$
    % zxShort/.style={anchor=center,minimum size=5mm, font={\footnotesize\boldmath}, shape=rectangle, rounded corners=2mm, inner sep=0.2mm, outer sep=-2mm, scale=0.8, draw=black},
    zxShort/.style={
      zxAllNodes,
      zxSpiders,
      minimum size=5mm,
      font={\footnotesize\boldmath},
      rounded rectangle,
      inner sep=0.0mm,
      scale=0.8,
    }, % negative outer sep would draw lines from inside...
    % Style for nodes that are bigger, like $\alpha+\beta$ or $(a\oplus b)\pi$
    zxLong/.style={zxShort, inner xsep=1.2mm},
    %%% Style that was supposed to chooses which style to apply depending on the input text
    %%% Can't find how to do.
    % zxChoose/.code={
    %   %% To fix
    % },
    %%%%%%%%%%% Style defined depending on above ones. Feel free to redefine them yourself.
    zxNoPhaseZ/.style={zxNoPhase,fill=colorZxZ},
    zxNoPhaseX/.style={zxNoPhase,fill=colorZxX},
    zxNoPhaseSmallZ/.style={zxNoPhaseSmall,fill=colorZxZ},
    zxNoPhaseSmallX/.style={zxNoPhaseSmall,fill=colorZxX},
    zxShortZ/.style={zxShort,fill=colorZxZ},
    zxShortX/.style={zxShort,fill=colorZxX},
    zxLongZ/.style={zxLong,fill=colorZxZ},
    zxLongX/.style={zxLong,fill=colorZxX},
    %%% Was supposed to automatically find the good style depending on content... Can't find how to do.
    % Styles zxLong{X/Z} zxNoPhase{X/Z} are automatically selected by \zxZ{...} and \zxX{...} commands
    % and zxShort is selected for fractions only like in \zxFracZ-{\pi}{4}
    % zxZ/.style={zxChoose={##1},fill=colorZxZ},
    % zxX/.style={zxChoose={##1},fill=colorZxX},
    % For Hadamard
    zxH/.style={
      zxAllNodes,
      outer sep=0pt,
      fill=colorZxH,
      draw,
      inner sep=0.6mm,
      minimum height=1.5mm,
      minimum width=1.5mm,
      shape=rectangle},
    zxHSmall/.style={zxH},
  },
  % Default style. Can be changed by user
  /zx/default style nodes/.style={
    /zx/styles/rounded style
  },
  % User can put here any additional property
  /zx/user overlay nodes/.style={
  },
  % Default wire style. Can be changed by user.
  /zx/default style wires/.style={
  },
  % User can add stuff in this style to improve wire styles
  /zx/user overlay wires/.style={
  },
  /zx/defaultEnv/.style={
    column sep=tiny,
    row sep=tiny,
    % center on the math axis
    baseline={([yshift=-axis_height]current bounding box.center)},
    % Fix 1-row diagram baseline
    1-row diagram/.style={%
      /tikz/baseline={([yshift=-axis_height]current bounding box.center)}%
    },
    % Load (thanks ".search also") our own style
    /tikz/every node/.style={%
      /zx/default style nodes,
      /zx/user overlay nodes,
    },
    every arrow/.style={%
      /zx/wires definition,
      /zx/default style wires,
      /zx/user overlay wires,
    },
  },
}

%%%%%%%%%%%%%%%%%%%%%%%%%%%%%%
%%%% Helper functions
%%%%%%%%%%%%%%%%%%%%%%%%%%%%%%

% Defines a "on layer=nameoflayer" style. TODO: check if better to move it in /zx/
% https://tex.stackexchange.com/questions/20425/z-level-in-tikz/20426#20426
\pgfkeys{%
  /tikz/on layer/.code={
    \def\tikz@path@do@at@end{\endpgfonlayer\endgroup\tikz@path@do@at@end}%
    \pgfonlayer{#1}\begingroup%
  }%
}

%%% Declare a symbol for a short minus (useful in fractions)
\DeclareMathSymbol{\zxMinus}{\mathbin}{AMSa}{"39} % Requires amssymb

%%% Create different kinds of dots...
%% https://tex.stackexchange.com/questions/617959
%% https://tex.stackexchange.com/questions/528774/excess-vertical-space-in-vdots/528775#528775
\DeclareRobustCommand\cvdotsAboveBaseline{%
  \vbox{\baselineskip4\p@ \lineskiplimit\z@%
    \hbox{.}\hbox{.}\hbox{.}}
}

\DeclareRobustCommand{\cvdotsCenterMathline}{%
  % vcenter is used to center the argument on the 'math axis', which is at half the height of an 'x', or about the position of a minus sign.
  \vcenter{\cvdotsAboveBaseline}%
}

\DeclareRobustCommand{\cvdotsCenterBaseline}{%
  \raisebox{-.5\height}{%
    $\cvdotsAboveBaseline$%
  }%
}

\DeclareRobustCommand{\chdots}{%
  \raisebox{-.5\height}{%
    \rotatebox{90}{% Maybe better options than rotatebox...
      $\cvdotsAboveBaseline$%
    }%
  }%
}

\DeclareRobustCommand{\cvdots}{\cvdotsCenterMathline}


%%%%%%%%%%%%%%%%%%%%%%%%%%%%%%%%%%%%%%%%%%%%%%%%%%%%%%%%%%%%%%%%%%%%%%%%%
%%% Practical macros to automatically choose appropriate style and arrows
%%%%%%%%%%%%%%%%%%%%%%%%%%%%%%%%%%%%%%%%%%%%%%%%%%%%%%%%%%%%%%%%%%%%%%%%%
% /!\ Warning: you should add {} at the end of all macros (except arrows)!
% Not using that may work for now, but it may break later...
% TODO: define them only in \zx environment.

% % Example: \leftManyDots{n}
% Useful to put on the left of a node like "n \vdots", linked to the next node.  Example: \leftManyDots{n}.
% First optional argument is scale of text, second is scale of =.
\NewExpandableDocumentCommand{\leftManyDots}{O{1}O{\zxScaleDots}m}{%
  |[zxNone+,inner xsep=0pt]| \scalebox{#1}{$#3$\,}\makebox[0pt][l]{\scalebox{#2}{$\cvdots$}} \ar[r,-s.,start anchor=north east] \ar[r,-s',start anchor=south east] \pgfmatrixnextcell%
}

% Useful to link two nodes and put a vdots in between.
\NewExpandableDocumentCommand{\middleManyDots}{}{%
  \ar[r,3 vdots] \ar[o',r] \ar[o.,r]%
}

% Like \leftManyDots but on the right. Do *not* create a new node, like in |[zxShortZ]| \alpha \rightManyDots{m}
\NewExpandableDocumentCommand{\rightManyDots}{O{1}O{\zxScaleDots}m}{%
  \ar[r,s'-,end anchor=north west] \ar[r,s.-,end anchor=south west] \pgfmatrixnextcell |[zxNone+,inner xsep=0pt]| \makebox[0pt][r]{\scalebox{#2}{$\cvdots$}}\scalebox{#1}{\,$#3$}
}

% A swap on one line... Practical mostly to gain space. Must be used with large nodes tough...
% \NewExpandableDocumentCommand{\OneLineSwap}{}{%
%   \ar[r,s,start anchor=south,end anchor=north] \ar[r,s,start anchor=north,end anchor=south]
% }


\NewExpandableDocumentCommand{\zxLoop}{O{90}O{20}O{}m}{%
  \ar[loop,in=#1-#2,out=#1+#2,looseness=8,min distance=3mm,#3]
}

\NewExpandableDocumentCommand{\zxLoopAboveDots}{O{20}O{}m}{%
  \ar[loop,in=90-#1,out=90+#1,looseness=8,min distance=3mm,"\cvdots" {scale=.6,anchor=north,yshift=-0.4mm},#2]
}

% Usage: node without any style, but may have space. Default is no space, \zxNone+{} is both horizontal
% and vertical, \zxNone-{} is only horizontal space, \zxNone|{} is only vertical space.
\NewExpandableDocumentCommand{\zxNone}{t+t-t|O{}m}{
  \IfBooleanTF{#1}{ % \zxNone+
    |[zxNone+,#4]| #5%
  }{
    \IfBooleanTF{#2}{ % \zxNone-
      |[zxNone-,#4]| #5%
    }{
      \IfBooleanTF{#3}{ % \zxNone
        |[zxNoneI,#4]| #5%
      }{% \zxNone
        |[zxNone,#4]| #5%
      }
    }
  }
}

% Cf \zxNone, but with larger space.
\NewExpandableDocumentCommand{\zxNoneDouble}{t+t-t|O{}m}{
  \IfBooleanTF{#1}{ % \zxNoneDouble+
    |[zxNoneDouble+,#4]| #5%
  }{
    \IfBooleanTF{#2}{ % \zxNoneDouble-
      |[zxNoneDouble-,#4]| #5%
    }{
      \IfBooleanTF{#3}{ % \zxNoneDouble
        |[zxNoneDoubleI,#4]| #5%
      }{% \zxNoneDouble
        |[zxNoneDouble,#4]| #5%
      }
    }
  }
}


\NewExpandableDocumentCommand{\zxX}{O{}m}{
  \ifblank{#2}{
    |[zxNoPhaseX,#1]| {}%
  }{%
    |[zxLongX,#1]| #2%
  }%
}

\NewExpandableDocumentCommand{\zxZ}{O{}m}{
  \ifblank{#2}{
    |[zxNoPhaseZ,#1]| {}%
  }{%
    |[zxLongZ,#1]| #2%
  }%
}

\NewExpandableDocumentCommand{\zxH}{O{}m}{
  |[zxH,#1]| {}%
}

% Use like: \zxFracX{\pi}{4} for positive values or for negative \zxFracX-{\pi}{4}
\NewExpandableDocumentCommand{\zxFracX}{t-mm}{
  |[zxShortX]| \IfBooleanTF{#1}{\zxMinus}{}\frac{#2}{#3}%
}

% Use like: \zxFracZ{\pi}{4} for positive values or for negative \zxFracZ-{\pi}{4}
\NewExpandableDocumentCommand{\zxFracZ}{t-mm}{
  |[zxShortZ]| \IfBooleanTF{#1}{\zxMinus}{}\frac{#2}{#3}%
}

\NewExpandableDocumentCommand{\zxEmptyDiagram}{}{
  |[zxEmptyDiagram]| {}%
}

% Quantikz has a bug which adds space automatically.
% https://tex.stackexchange.com/questions/618330
% Fixing that by copying the original (unpatched) functions, and reusing them later.
% Warning: you must load this package **before** quantikz otherwise the fix will not work.
\let\tikzcd@@originalCopyZx\tikzcd@
\let\endtikzcd@originalCopyZx\endtikzcd

%%%%% Main environment \begin{ZX}...\end{ZX}
\NewDocumentEnvironment{ZX}{O{}}{%
  \bgroup%
  % Add a switch in case someone really wants the current tikzcd version:
  \ifdefined\doNotPatchQuantikz% Do not patch tikzcd.
  \else% Restore locally original tikzcd.
    \let\tikzcd@\tikzcd@@originalCopyZx%
    \let\endtikzcd\endtikzcd@originalCopyZx%
  \fi%
  \pgfsetlayers{background,main} % Layers are defined locally to avoid to disturb other drawings
  \begin{tikzcd}[%
    /zx/defaultEnv,%
    #1]%
  }{\end{tikzcd}\egroup}

%%%%% Shortcut macro \zx{...} equivalent to \begin{ZX}...\end{ZX}
\newcommand\zx{%
  \begingroup% To avoid ampersand issues https://tex.stackexchange.com/a/611535/116348
  \NewDocumentCommand{\tmpZX}{O{}+m}{%
    \endgroup%
    \begin{ZX}[##1]%
      ##2%
    \end{ZX}%
  }%
  \catcode`&=13%
  \tmpZX%
}

%%%%%%%%%%%%%%%%%%%%%%%%%%%%%%
%%% Old code that tried to automatically find if zxShort or zxLong should be used...
%%% Now using a special command for fractions (easier to code, and more customizable)
%%%%%%%%%%%%%%%%%%%%%%%%%%%%%%
\newsavebox\zx@box % Temporary box to compute height/width/depth

\newlength{\zxMaxDepthPlusHeight}\setlength{\zxMaxDepthPlusHeight}{2em}
\def\zxMaxRatio{1.3} % Ratio width/(height+depth)

\NewExpandableDocumentCommand{\zxChooseStyle}{mmmm}{%
  % #1=text,#2=empty style,#3=short style,#4=long style
  \savebox\zx@box{#1}%
  % Check if width is 0pt:
  \ifdimcomp{\wd\zx@box}{=}{0pt}{% Return empty style if box is empty
    #2%
  }{% Else compute size of thext
    % Check if height+depth < zxMaxDepthPlusHeight to see if short style applies
    \ifdimcomp{\dimexpr\dp\zx@box+\ht\zx@box\relax}{<}{\zxMaxDepthPlusHeight}{%
      % Check if width < ratio*(height+depth) to see if short style applies
      \ifdimcomp{\wd\zx@box}{<}{\dimexpr \zxMaxRatio\ht\zx@box + \zxMaxRatio\dp\zx@box\relax}{%
        #3% Short style is used
      }{ % Else
        #4% Long style is used
      } %
    }{
      #4% Long style is used
    }
  }%
}

\makeatother
% Loads the great package that produces tikz-like manual (see also tikzcd for examples)
% Copyright 2019 by Till Tantau
%
% This file may be distributed and/or modified
%
% 1. under the LaTeX Project Public License and/or
% 2. under the GNU Free Documentation License.
%
% See the file doc/generic/pgf/licenses/LICENSE for more details.

% $Header$


\newcount\pgfmanualtargetcount

\colorlet{examplefill}{yellow!80!black}
\definecolor{graphicbackground}{rgb}{0.96,0.96,0.8}
\definecolor{codebackground}{rgb}{0.9,0.9,1}
\definecolor{animationgraphicbackground}{rgb}{0.96,0.96,0.8}

\newenvironment{pgfmanualentry}{\list{}{\leftmargin=2em\itemindent-\leftmargin\def\makelabel##1{\hss##1}}}{\endlist}
\newcounter{pgfmanualentry}
\newcommand\pgfmanualentryheadline[1]{%
  \itemsep=0pt\parskip=0pt{\raggedright\item\refstepcounter{pgfmanualentry}\strut{#1}\par}\topsep=0pt}
\newcommand\pgfmanualbody{\parskip3pt}

\let\origtexttt=\texttt
\def\texttt#1{{\def\textunderscore{\char`\_}\def\textbraceleft{\char`\{}\def\textbraceright{\char`\}}\origtexttt{#1}}}
\def\exclamationmarktext{!}
\def\atmarktext{@}

{
  \catcode`\|=12
  \gdef\pgfmanualnormalbar{|}
  \catcode`\|=13
  \AtBeginDocument{\gdef|{\ifmmode\pgfmanualnormalbar\else\expandafter\verb\expandafter|\fi}}
}



\newenvironment{pgflayout}[1]{
  \begin{pgfmanualentry}
    \pgfmanualentryheadline{%
      \pgfmanualpdflabel{#1}{}%
      \texttt{\string\pgfpagesuselayout\char`\{\declare{#1}\char`\}}\oarg{options}%
    }
    \index{#1@\protect\texttt{#1} layout}%
    \index{Page layouts!#1@\protect\texttt{#1}}%
    \pgfmanualbody
}
{
  \end{pgfmanualentry}
}


\newenvironment{sysanimateattribute}[1]{
  \begin{pgfmanualentry}
    \pgfmanualentryheadline{%
      \pgfmanualpdflabel{#1}{}%
      \texttt{\string\pgfsysanimate\char`\{\declare{#1}\char`\}}%
    }
    \index{#1@\protect\texttt{#1} system layer animation attribute}%
    \index{Animation attributes (system layer)!#1@\protect\texttt{#1}}%
    \pgfmanualbody
}
{
  \end{pgfmanualentry}
}


\newenvironment{animateattribute}[1]{
  \begin{pgfmanualentry}
    \pgfmanualentryheadline{%
      \pgfmanualpdflabel{#1}{}%
      \texttt{\string\pgfanimateattribute\char`\{\declare{#1}\char`\}\marg{options}}%
    }
    \index{#1@\protect\texttt{#1} basic layer animation attribute}%
    \index{Animation attributes (basic layer)!#1@\protect\texttt{#1}}%
    \pgfmanualbody
}
{
  \end{pgfmanualentry}
}


\newenvironment{tikzanimateattribute}[1]{
  \begin{pgfmanualentry}
    \pgfmanualentryheadline{%
      \foreach \attr in{#1} {\expandafter\pgfmanualpdflabel\expandafter{\attr}{}}%
      \textbf{Animation attribute} \foreach \attr[count=\i]
      in{#1}{{\ifnum\i>1 \textbf,\fi} \texttt{:\declare{\attr}}}%
    }
    \foreach\attr in{#1}{%
      \edef\indexcall{%
        \noexpand\index{\attr@\noexpand\protect\noexpand\texttt{\attr} animation attribute}%
        \noexpand\index{Animation attributes!\attr@\noexpand\protect\noexpand\texttt{\attr}}%
      }%
      \indexcall%
    }%
    \pgfmanualbody
}
{
  \end{pgfmanualentry}
}


\newenvironment{command}[1]{
  \begin{pgfmanualentry}
    \extractcommand#1\@@
    \pgfmanualbody
}
{
  \end{pgfmanualentry}
}

\makeatletter

\def\includeluadocumentationof#1{
  \directlua{require 'pgf.manual.DocumentParser'}
  \directlua{pgf.manual.DocumentParser.include '#1'}
}

\newenvironment{luageneric}[4]{
  \pgfmanualentry
    \pgfmanualentryheadline{#4 \texttt{#1\declare{#2}}#3}
    \index{#2@\protect\texttt{#2} (Lua)}%
    \def\temp{#1}
    \ifx\temp\pgfutil@empty\else
      \index{#1@\protect\texttt{#1}!#2@\protect\texttt{#2} (Lua)}%
    \fi
  \pgfmanualbody
}{\endpgfmanualentry}

\newenvironment{luatable}[3]{
  \medskip
  \luageneric{#1}{#2}{ (declared in \texttt{#3})}{\textbf{Lua table}}
}{\endluageneric}

\newenvironment{luafield}[1]{
  \pgfmanualentry
    \pgfmanualentryheadline{Field \texttt{\declare{#1}}}
  \pgfmanualbody
}{\endpgfmanualentry}


\newenvironment{lualibrary}[1]{
  \pgfmanualentry
  \pgfmanualentryheadline{%
    \pgfmanualpdflabel{#1}{}%
    \textbf{Graph Drawing Library} \texttt{\declare{#1}}%
  }
    \index{#1@\protect\texttt{#1} graph drawing library}%
    \index{Libraries!#1@\protect\texttt{#1}}%
    \index{Graph drawing libraries!#1@\protect\texttt{#1}}%
    \vskip.25em
    {\ttfamily\char`\\usegdlibrary\char`\{\declare{#1}\char`\}\space\space \char`\%\space\space  \LaTeX\space and plain \TeX}\\
    {\ttfamily\char`\\usegdlibrary[\declare{#1}]\space \char`\%\space\space Con\TeX t}\smallskip\par
    \pgfmanualbody
}{\endpgfmanualentry}

\newenvironment{luadeclare}[4]{
  \pgfmanualentry
  \def\manual@temp@default{#3}%
  \def\manual@temp@initial{#4}%
  \def\manual@temp@{#3#4}%
  \pgfmanualentryheadline{%
    \pgfmanualpdflabel{#1}{}%
    {\ttfamily/graph
      drawing/\declare{#1}\opt{=}}\opt{#2}\hfill%
    \ifx\manual@temp@\pgfutil@empty\else%
    (\ifx\manual@temp@default\pgfutil@empty\else%
    default {\ttfamily #3}\ifx\manual@temp@initial\pgfutil@empty\else, \fi%
    \fi%
    \ifx\manual@temp@initial\pgfutil@empty\else%
    initially {\ttfamily #4}%
    \fi%
    )\fi%
  }%
  \index{#1@\protect\texttt{#1} key}%
  \pgfmanualbody
  \gdef\myname{#1}%
%  \keyalias{tikz}
%  \keyalias{tikz/graphs}
}{\endpgfmanualentry}

\newenvironment{luadeclarestyle}[4]{
  \pgfmanualentry
  \def\manual@temp@para{#2}%
  \def\manual@temp@default{#3}%
  \def\manual@temp@initial{#4}%
  \def\manual@temp@{#3#4}%
  \pgfmanualentryheadline{%
    \pgfmanualpdflabel{#1}{}%
    {\ttfamily/graph drawing/\declare{#1}}\ifx\manual@temp@para\pgfutil@empty\else\opt{\texttt=}\opt{#2}\fi\hfill%
    (style\ifx\manual@temp@\pgfutil@empty\else, %
    \ifx\manual@temp@default\pgfutil@empty\else%
    default {\ttfamily #3}\ifx\manual@temp@initial\pgfutil@empty\else, \fi%
    \fi%
    \ifx\manual@temp@initial\pgfutil@empty\else%
    initially {\ttfamily #4}%
    \fi%
    \fi)%
  }%
  \index{#1@\protect\texttt{#1} key}%
  \pgfmanualbody%
  \gdef\myname{#1}%
%  \keyalias{tikz}
%  \keyalias{tikz/graphs}
}{\endpgfmanualentry}

\newenvironment{luanamespace}[2]{
  \luageneric{#1}{#2}{}{\textbf{Lua namespace}}
}{\endluageneric}

\newenvironment{luafiledescription}[1]{}{}

\newenvironment{luacommand}[4]{
  \hypertarget{pgf/lua/#1}{\luageneric{#2}{#3}{\texttt{(#4)}}{\texttt{function}}}
}{\endluageneric}

\newenvironment{luaparameters}{\par\emph{Parameters:}%
  \parametercount=0\relax%
  \let\item=\parameteritem%
  \let\list=\restorelist%
}
{\par
}

\newenvironment{luareturns}{\par\emph{Returns:}%
  \parametercount=0\relax%
  \let\item=\parameteritem%
  \let\list=\restorelist%
}
{\par
}

\newcount\parametercount

\newenvironment{parameterdescription}{\unskip%
  \parametercount=0\relax%
  \let\item=\parameteritem%
  \let\list=\restorelist%
}
{\par
}
\let\saveditemcommand=\item
\let\savedlistcommand=\list
\def\denselist#1#2{\savedlistcommand{#1}{#2}\parskip0pt\itemsep0pt}
\def\restorelist{\let\item=\saveditemcommand\denselist}
\def\parameteritem{\pgfutil@ifnextchar[\parameteritem@{}}%}
\def\parameteritem@[#1]{\advance\parametercount by1\relax\hskip0.15em plus 1em\emph{\the\parametercount.}\kern1ex\def\test{#1}\ifx\test\pgfutil@empty\else#1\kern.5em\fi}

\newenvironment{commandlist}[1]{%
  \begin{pgfmanualentry}
  \foreach \xx in {#1} {%
    \expandafter\extractcommand\xx\@@
  }%
  \pgfmanualbody
}{%
  \end{pgfmanualentry}
}%

% \begin{internallist}[register]{\pgf@xa}
% \end{internallist}
%
% \begin{internallist}[register]{\pgf@xa,\pgf@xb}
% \end{internallist}
\newenvironment{internallist}[2][register]{%
  \begin{pgfmanualentry}
  \foreach \xx in {#2} {%
    \expandafter\extractinternalcommand\expandafter{\xx}{#1}%
  }%
  \pgfmanualbody
}{%
  \end{pgfmanualentry}
}%
\def\extractinternalcommand#1#2{%
  \removeats{#1}%
  \pgfmanualentryheadline{%
    \pgfmanualpdflabel{\textbackslash\strippedat}{}%
    Internal #2 \declare{\texttt{\string#1}}}%
  \index{Internals!\strippedat @\protect\myprintocmmand{\strippedat}}%
  \index{\strippedat @\protect\myprintocmmand{\strippedat}}%
}

%% MW: START MATH MACROS
\def\mvar#1{{\ifmmode\textrm{\textit{#1}}\else\rmfamily\textit{#1}\fi}}

\makeatletter

\def\extractmathfunctionname#1{\extractmathfunctionname@#1(,)\tmpa\tmpb}
\def\extractmathfunctionname@#1(#2)#3\tmpb{\def\mathname{#1}}

\makeatother
  
\newenvironment{math-function}[1]{
  \def\mathdefaultname{#1}
  \extractmathfunctionname{#1}
  \edef\mathurl{{math:\mathname}}\expandafter\hypertarget\expandafter{\mathurl}{}%
  \begin{pgfmanualentry}
    \pgfmanualentryheadline{\texttt{#1}}%
    \index{\mathname @\protect\texttt{\mathname} math function}%
    \index{Math functions!\mathname @\protect\texttt{\mathname}}%
    \pgfmanualbody
}
{
  \end{pgfmanualentry}
}

\def\pgfmanualemptytext{}
\def\pgfmanualvbarvbar{\char`\|\char`\|}

\newenvironment{math-operator}[4][]{%
  \begin{pgfmanualentry}
  \csname math#3operator\endcsname{#2}{#4}
  \def\mathtest{#4}%
  \ifx\mathtest\pgfmanualemptytext%
    \def\mathtype{(#3 operator)}
  \else%
    \def\mathtype{(#3 operator; uses the \texttt{#4} function)}
  \fi%
  \pgfmanualentryheadline{\mathexample\hfill\mathtype}%
  \def\mathtest{#1}%
  \ifx\mathtest\pgfmanualemptytext%
    \index{#2@\protect\texttt{#2} #3 math operator}%  
    \index{Math operators!#2@\protect\texttt{#2}}%
  \fi%
  \pgfmanualbody
}
{\end{pgfmanualentry}}

\newenvironment{math-operators}[5][]{%
  \begin{pgfmanualentry}
  \csname math#4operator\endcsname{#2}{#3}
  \def\mathtest{#5}%
  \ifx\mathtest\pgfmanualemptytext%
    \def\mathtype{(#4 operators)}
  \else%
    \def\mathtype{(#4 operators; use the \texttt{#5} function)}
  \fi%
  \pgfmanualentryheadline{\mathexample\hfill\mathtype}%
  \def\mathtest{#1}%
  \ifx\mathtest\pgfmanualemptytext%
    \index{#2#3@\protect\texttt{#2\protect\ #3} #4 math operators}% 
    \index{Math operators!#2#3@\protect\texttt{#2\protect\ #3}}%
  \fi%
  \pgfmanualbody
}
{\end{pgfmanualentry}}

\def\mathinfixoperator#1#2{%
  \def\mathoperator{\texttt{#1}}%
  \def\mathexample{\mvar{x}\space\texttt{#1}\space\mvar{y}}%
}

\def\mathprefixoperator#1#2{%
  \def\mathoperator{\texttt{#1}}%
  \def\mathexample{\texttt{#1}\mvar{x}}%
}

\def\mathpostfixoperator#1#2{%
  \def\mathoperator{\texttt{#1}}
  \def\mathexample{\mvar{x}\texttt{#1}}%
}

\def\mathgroupoperator#1#2{%
  \def\mathoperator{\texttt{#1\ #2}}%
  \def\mathexample{\texttt{#1}\mvar{x}\texttt{#2}}%
}

\expandafter\let\csname matharray accessoperator\endcsname=\mathgroupoperator
\expandafter\let\csname matharrayoperator\endcsname=\mathgroupoperator

\def\mathconditionaloperator#1#2{%
  \def\mathoperator{#1\space#2}
  \def\mathexample{\mvar{x}\ \texttt{#1}\ \mvar{y}\ {\texttt{#2}}\ \mvar{z}}
}

\newcommand\mathcommand[1][\mathdefaultname]{%
  \expandafter\makemathcommand#1(\empty)\stop%
  \expandafter\extractcommand\mathcommandname\@@%
  \medskip
}
\makeatletter

\def\makemathcommand#1(#2)#3\stop{%
  \expandafter\def\expandafter\mathcommandname\expandafter{\csname pgfmath#1\endcsname}%
  \ifx#2\empty%
  \else%
    \@makemathcommand#2,\stop,
  \fi}
\def\@makemathcommand#1,{%
  \ifx#1\stop%
  \else%
    \expandafter\def\expandafter\mathcommandname\expandafter{\mathcommandname{\ttfamily\char`\{#1\char`\}}}%
    \expandafter\@makemathcommand%
  \fi}
\makeatother

\def\calcname{\textsc{calc}}

\newenvironment{math-keyword}[1]{
  \extracttikzmathkeyword#1@
  \begin{pgfmanualentry}
    \pgfmanualentryheadline{\texttt{\color{red}\mathname}\mathrest}%
    \index{\mathname @\protect\texttt{\mathname} tikz math function}%
    \index{TikZ math functions!\mathname @\protect\texttt{\mathname}}%
    \pgfmanualbody
}
{
  \end{pgfmanualentry}
}

\def\extracttikzmathkeyword#1#2@{%
  \def\mathname{#1}%
  \def\mathrest{#2}%
}

%% MW: END MATH MACROS


\def\extractcommand#1#2\@@{%
  \removeats{#1}%
  \pgfmanualentryheadline{%
    \pgfmanualpdflabel{\textbackslash\strippedat}{}%
    \declare{\expandafter\texttt\expandafter{\string#1}}#2%
  }%
  \index{\strippedat @\protect\myprintocmmand{\strippedat}}
}

\def\luaextractcommand#1#2\relax{%
  \declare{\texttt{\string#1}}#2\par%
%  \removeats{#1}%
 % \index{\strippedat @\protect\myprintocmmand{\strippedat}}
 % \pgfmanualpdflabel{\textbackslash\strippedat}{}%
}


% \begin{environment}{{name}\marg{arguments}}
\renewenvironment{environment}[1]{
  \begin{pgfmanualentry}
    \extractenvironement#1\@@
    \pgfmanualbody
}
{
  \end{pgfmanualentry}
}

\def\extractenvironement#1#2\@@{%
  \pgfmanualentryheadline{%
    \pgfmanualpdflabel{#1}{}%
    {\ttfamily\char`\\begin\char`\{\declare{#1}\char`\}}#2%
  }%
  \pgfmanualentryheadline{{\ttfamily\ \ }\meta{environment contents}}%
  \pgfmanualentryheadline{{\ttfamily\char`\\end\char`\{\declare{#1}\char`\}}}%
  \index{#1@\protect\texttt{#1} environment}%
  \index{Environments!#1@\protect\texttt{#1}}
}


\newenvironment{plainenvironment}[1]{
  \begin{pgfmanualentry}
    \extractplainenvironement#1\@@
    \pgfmanualbody
}
{
  \end{pgfmanualentry}
}

\def\extractplainenvironement#1#2\@@{%
  \pgfmanualentryheadline{{\ttfamily\declare{\char`\\#1}}#2}%
  \pgfmanualentryheadline{{\ttfamily\ \ }\meta{environment contents}}%
  \pgfmanualentryheadline{{\ttfamily\declare{\char`\\end#1}}}%
  \index{#1@\protect\texttt{#1} environment}%
  \index{Environments!#1@\protect\texttt{#1}}%
}


\newenvironment{contextenvironment}[1]{
  \begin{pgfmanualentry}
    \extractcontextenvironement#1\@@
    \pgfmanualbody
}
{
  \end{pgfmanualentry}
}

\def\extractcontextenvironement#1#2\@@{%
  \pgfmanualentryheadline{{\ttfamily\declare{\char`\\start#1}}#2}%
  \pgfmanualentryheadline{{\ttfamily\ \ }\meta{environment contents}}%
  \pgfmanualentryheadline{{\ttfamily\declare{\char`\\stop#1}}}%
  \index{#1@\protect\texttt{#1} environment}%
  \index{Environments!#1@\protect\texttt{#1}}}


\newenvironment{shape}[1]{
  \begin{pgfmanualentry}
    \pgfmanualentryheadline{%
      \pgfmanualpdflabel{#1}{}%
      \textbf{Shape} {\ttfamily\declare{#1}}%
    }%
    \index{#1@\protect\texttt{#1} shape}%
    \index{Shapes!#1@\protect\texttt{#1}}
    \pgfmanualbody
}
{
  \end{pgfmanualentry}
}

\newenvironment{pictype}[2]{
  \begin{pgfmanualentry}
    \pgfmanualentryheadline{%
      \pgfmanualpdflabel{#1}{}%
      \textbf{Pic type} {\ttfamily\declare{#1}#2}%
    }%
    \index{#1@\protect\texttt{#1} pic type}%
    \index{Pic Types!#1@\protect\texttt{#1}}
    \pgfmanualbody
}
{
  \end{pgfmanualentry}
}

\newenvironment{shading}[1]{
  \begin{pgfmanualentry}
    \pgfmanualentryheadline{%
      \pgfmanualpdflabel{#1}{}%
      \textbf{Shading} {\ttfamily\declare{#1}}}%
    \index{#1@\protect\texttt{#1} shading}%
    \index{Shadings!#1@\protect\texttt{#1}}
    \pgfmanualbody
}
{
  \end{pgfmanualentry}
}


\newenvironment{graph}[1]{
  \begin{pgfmanualentry}
    \pgfmanualentryheadline{%
      \pgfmanualpdflabel{#1}{}%
      \textbf{Graph} {\ttfamily\declare{#1}}}%
    \index{#1@\protect\texttt{#1} graph}%
    \index{Graphs!#1@\protect\texttt{#1}}
    \pgfmanualbody
}
{
  \end{pgfmanualentry}
}

\newenvironment{gdalgorithm}[2]{
  \begin{pgfmanualentry}
    \pgfmanualentryheadline{%
      \pgfmanualpdflabel{#1}{}%
      \textbf{Layout} {\ttfamily/graph drawing/\declare{#1}\opt{=}}\opt{\meta{options}}}%
    \index{#1@\protect\texttt{#1} layout}%
    \index{Layouts!#1@\protect\texttt{#1}}%
    \foreach \algo in {#2}
    {\edef\marshal{\noexpand\index{#2@\noexpand\protect\noexpand\texttt{#2} algorithm}}\marshal}%
    \index{Graph drawing layouts!#1@\protect\texttt{#1}}
    \item{\small alias {\ttfamily/tikz/#1}}\par
    \item{\small alias {\ttfamily/tikz/graphs/#1}}\par
    \item{\small Employs {\ttfamily algorithm=#2}}\par
    \pgfmanualbody
}
{
  \end{pgfmanualentry}
}

\newenvironment{dataformat}[1]{
  \begin{pgfmanualentry}
    \pgfmanualentryheadline{%
      \pgfmanualpdflabel{#1}{}%
      \textbf{Format} {\ttfamily\declare{#1}}}%
    \index{#1@\protect\texttt{#1} format}%
    \index{Formats!#1@\protect\texttt{#1}}
    \pgfmanualbody
}
{
  \end{pgfmanualentry}
}

\newenvironment{stylesheet}[1]{
  \begin{pgfmanualentry}
    \pgfmanualentryheadline{%
      \pgfmanualpdflabel{#1}{}%
      \textbf{Style sheet} {\ttfamily\declare{#1}}}%
    \index{#1@\protect\texttt{#1} style sheet}%
    \index{Style sheets!#1@\protect\texttt{#1}}
    \pgfmanualbody
}
{
  \end{pgfmanualentry}
}

\newenvironment{handler}[1]{
  \begin{pgfmanualentry}
    \extracthandler#1\@nil%
    \pgfmanualbody
}
{
  \end{pgfmanualentry}
}

\def\gobble#1{}
\def\extracthandler#1#2\@nil{%
  \pgfmanualentryheadline{%
    \pgfmanualpdflabel{/handlers/#1}{}%
    \textbf{Key handler} \meta{key}{\ttfamily/\declare{#1}}#2}%
  \index{\gobble#1@\protect\texttt{#1} handler}%
  \index{Key handlers!#1@\protect\texttt{#1}}
}


\makeatletter


\newenvironment{stylekey}[1]{
  \begin{pgfmanualentry}
    \def\extrakeytext{style, }
    \extractkey#1\@nil%
    \pgfmanualbody
}
{
  \end{pgfmanualentry}
}

\def\choicesep{$\vert$}%
\def\choicearg#1{\texttt{#1}}

\newif\iffirstchoice

% \mchoice{choice1,choice2,choice3}
\newcommand\mchoice[1]{%
  \begingroup
  \firstchoicetrue
  \foreach \mchoice@ in {#1} {%
    \iffirstchoice
      \global\firstchoicefalse
    \else
      \choicesep
    \fi
    \choicearg{\mchoice@}%
  }%
  \endgroup
}%

% \begin{key}{/path/x=value}
% \begin{key}{/path/x=value (initially XXX)}
% \begin{key}{/path/x=value (default XXX)}
\newenvironment{key}[1]{
  \begin{pgfmanualentry}
    \def\extrakeytext{}
    %\def\altpath{\emph{\color{gray}or}}%
    \extractkey#1\@nil%
    \pgfmanualbody
}
{
  \end{pgfmanualentry}
}

% \insertpathifneeded{a key}{/pgf} -> assign mykey={/pgf/a key}
% \insertpathifneeded{/tikz/a key}{/pgf} -> assign mykey={/tikz/a key}
%
% #1: the key
% #2: a default path (or empty)
\def\insertpathifneeded#1#2{%
  \def\insertpathifneeded@@{#2}%
  \ifx\insertpathifneeded@@\empty
    \def\mykey{#1}%
  \else
    \insertpathifneeded@#2\@nil
    \ifpgfutil@in@
      \def\mykey{#2/#1}%
    \else
      \def\mykey{#1}%
    \fi
  \fi
}%
\def\insertpathifneeded@#1#2\@nil{%
  \def\insertpathifneeded@@{#1}%
  \def\insertpathifneeded@@@{/}%
  \ifx\insertpathifneeded@@\insertpathifneeded@@@
    \pgfutil@in@true
  \else
    \pgfutil@in@false
  \fi
}%

% \begin{keylist}[default path]
%   {/path/option 1=value,/path/option 2=value2}
% \end{keylist}
\newenvironment{keylist}[2][]{%
  \begin{pgfmanualentry}
    \def\extrakeytext{}%
  \foreach \xx in {#2} {%
    \expandafter\insertpathifneeded\expandafter{\xx}{#1}%
    \expandafter\extractkey\mykey\@nil%
  }%
  \pgfmanualbody
}{%
  \end{pgfmanualentry}
}%

\def\extractkey#1\@nil{%
  \pgfutil@in@={#1}%
  \ifpgfutil@in@%
    \extractkeyequal#1\@nil
  \else%
    \pgfutil@in@{(initial}{#1}%
    \ifpgfutil@in@%
      \extractequalinitial#1\@nil%
    \else
      \pgfmanualentryheadline{%
      \def\mykey{#1}%
      \def\mypath{}%
      \gdef\myname{}%
      \firsttimetrue%
      \pgfmanualdecomposecount=0\relax%
      \decompose#1/\nil%
        {\ttfamily\declare{#1}}\hfill(\extrakeytext no value)}%
    \fi
  \fi%
}

\def\extractkeyequal#1=#2\@nil{%
  \pgfutil@in@{(default}{#2}%
  \ifpgfutil@in@%
    \extractdefault{#1}#2\@nil%
  \else%
    \pgfutil@in@{(initial}{#2}%
    \ifpgfutil@in@%
      \extractinitial{#1}#2\@nil%
    \else
      \pgfmanualentryheadline{%
        \def\mykey{#1}%
        \def\mypath{}%
        \gdef\myname{}%
        \firsttimetrue%
        \pgfmanualdecomposecount=0\relax%
        \decompose#1/\nil%
        {\ttfamily\declare{#1}=}#2\hfill(\extrakeytext no default)}%
    \fi%
  \fi%
}

\def\extractdefault#1#2(default #3)\@nil{%
  \pgfmanualentryheadline{%
    \def\mykey{#1}%
    \def\mypath{}%
    \gdef\myname{}%
    \firsttimetrue%
    \pgfmanualdecomposecount=0\relax%
    \decompose#1/\nil%
    {\ttfamily\declare{#1}\opt{=}}\opt{#2}\hfill (\extrakeytext default {\ttfamily#3})}%
}

\def\extractinitial#1#2(initially #3)\@nil{%
  \pgfmanualentryheadline{%
    \def\mykey{#1}%
    \def\mypath{}%
    \gdef\myname{}%
    \firsttimetrue%
    \pgfmanualdecomposecount=0\relax%
    \decompose#1/\nil%
    {\ttfamily\declare{#1}=}#2\hfill (\extrakeytext no default, initially {\ttfamily#3})}%
}

\def\extractequalinitial#1 (initially #2)\@nil{%
  \pgfmanualentryheadline{%
    \def\mykey{#1}%
    \def\mypath{}%
    \gdef\myname{}%
    \firsttimetrue%
    \pgfmanualdecomposecount=0\relax%
    \decompose#1/\nil%
    {\ttfamily\declare{#1}}\hfill (\extrakeytext initially {\ttfamily#2})}%
}

% Introduces a key alias '/#1/<name of current key>'
% to be used inside of \begin{key} ... \end{key}
\def\keyalias#1{\vspace{-3pt}\item{\small alias {\ttfamily/#1/\myname}}\vspace{-2pt}\par
  \pgfmanualpdflabel{/#1/\myname}{}%
}

\newif\iffirsttime
\newcount\pgfmanualdecomposecount

\makeatother

\def\decompose/#1/#2\nil{%
  \def\test{#2}%
  \ifx\test\empty%
    % aha.
    \index{#1@\protect\texttt{#1} key}%
    \index{\mypath#1@\protect\texttt{#1}}%
    \gdef\myname{#1}%
    \pgfmanualpdflabel{#1}{}
  \else%
    \advance\pgfmanualdecomposecount by1\relax%
    \ifnum\pgfmanualdecomposecount>2\relax%
      \decomposetoodeep#1/#2\nil%
    \else%
      \iffirsttime%
        \begingroup%  
          % also make a pdf link anchor with full key path.
          \def\hyperlabelwithoutslash##1/\nil{%
            \pgfmanualpdflabel{##1}{}%
          }%
          \hyperlabelwithoutslash/#1/#2\nil%
        \endgroup%
        \def\mypath{#1@\protect\texttt{/#1/}!}%
        \firsttimefalse%
      \else%
        \expandafter\def\expandafter\mypath\expandafter{\mypath#1@\protect\texttt{#1/}!}%
      \fi%
      \def\firsttime{}%
      \decompose/#2\nil%
    \fi%
  \fi%
}

\def\decomposetoodeep#1/#2/\nil{%
  % avoid too-deep nesting in index
  \index{#1/#2@\protect\texttt{#1/#2} key}%
  \index{\mypath#1/#2@\protect\texttt{#1/#2}}%
  \decomposefindlast/#1/#2/\nil%
}
\makeatletter
\def\decomposefindlast/#1/#2\nil{%
  \def\test{#2}%
  \ifx\test\pgfutil@empty%
    \gdef\myname{#1}%
  \else%
    \decomposefindlast/#2\nil%
  \fi%
}
\makeatother
\def\indexkey#1{%
  \def\mypath{}%
  \decompose#1/\nil%
}

\newenvironment{predefinedmethod}[1]{
  \begin{pgfmanualentry}
    \extractpredefinedmethod#1\@nil
    \pgfmanualbody
}
{
  \end{pgfmanualentry}
}
\def\extractpredefinedmethod#1(#2)\@nil{%
  \pgfmanualentryheadline{%
    \pgfmanualpdflabel{#1}{}%
    Method \declare{\ttfamily #1}\texttt(#2\texttt) \hfill(predefined for all classes)}
  \index{#1@\protect\texttt{#1} method}%
  \index{Methods!#1@\protect\texttt{#1}}
}


\newenvironment{ooclass}[1]{
  \begin{pgfmanualentry}
    \def\currentclass{#1}
    \pgfmanualentryheadline{%
      \pgfmanualpdflabel{#1}{}%
      \textbf{Class} \declare{\texttt{#1}}}
    \index{#1@\protect\texttt{#1} class}%
    \index{Class #1@Class \protect\texttt{#1}}%
    \index{Classes!#1@\protect\texttt{#1}}
    \pgfmanualbody
}
{
  \end{pgfmanualentry}
}

\newenvironment{method}[1]{
  \begin{pgfmanualentry}
    \extractmethod#1\@nil
    \pgfmanualbody
}
{
  \end{pgfmanualentry}
}
\def\extractmethod#1(#2)\@nil{%
  \def\test{#1}
  \ifx\test\currentclass
    \pgfmanualentryheadline{%
      \pgfmanualpdflabel{#1}{}%
      Constructor \declare{\ttfamily #1}\texttt(#2\texttt)}
  \else
    \pgfmanualentryheadline{%
      \pgfmanualpdflabel{#1}{}%
      Method \declare{\ttfamily #1}\texttt(#2\texttt)}
  \fi
  \index{#1@\protect\texttt{#1} method}%
  \index{Methods!#1@\protect\texttt{#1}}
  \index{Class \currentclass!#1@\protect\texttt{#1}}%
}

\newenvironment{classattribute}[1]{
  \begin{pgfmanualentry}
    \extractattribute#1\@nil
    \pgfmanualbody
}
{
  \end{pgfmanualentry}
}
\def\extractattribute#1=#2;\@nil{%
  \def\test{#2}%
  \ifx\test\@empty
    \pgfmanualentryheadline{%
      \pgfmanualpdflabel{#1}{}%
      Private attribute \declare{\ttfamily #1} \hfill (initially empty)}
  \else
    \pgfmanualentryheadline{%
      \pgfmanualpdflabel{#1}{}%
      Private attribute \declare{\ttfamily #1} \hfill (initially {\ttfamily #2})}
  \fi
  \index{#1@\protect\texttt{#1} attribute}%
  \index{Attributes!#1@\protect\texttt{#1}}
  \index{Class \currentclass!#1@\protect\texttt{#1}}%
}



\newenvironment{predefinednode}[1]{
  \begin{pgfmanualentry}
    \pgfmanualentryheadline{%
      \pgfmanualpdflabel{#1}{}%
      \textbf{Predefined node} {\ttfamily\declare{#1}}}%
    \index{#1@\protect\texttt{#1} node}%
    \index{Predefined node!#1@\protect\texttt{#1}}
    \pgfmanualbody
}
{
  \end{pgfmanualentry}
}

\newenvironment{coordinatesystem}[1]{
  \begin{pgfmanualentry}
    \pgfmanualentryheadline{%
      \pgfmanualpdflabel{#1}{}%
      \textbf{Coordinate system} {\ttfamily\declare{#1}}}%
    \index{#1@\protect\texttt{#1} coordinate system}%
    \index{Coordinate systems!#1@\protect\texttt{#1}}
    \pgfmanualbody
}
{
  \end{pgfmanualentry}
}

\newenvironment{snake}[1]{
  \begin{pgfmanualentry}
    \pgfmanualentryheadline{\textbf{Snake} {\ttfamily\declare{#1}}}%
    \index{#1@\protect\texttt{#1} snake}%
    \index{Snakes!#1@\protect\texttt{#1}}
    \pgfmanualbody
}
{
  \end{pgfmanualentry}
}

\newenvironment{decoration}[1]{
  \begin{pgfmanualentry}
    \pgfmanualentryheadline{\textbf{Decoration} {\ttfamily\declare{#1}}}%
    \index{#1@\protect\texttt{#1} decoration}%
    \index{Decorations!#1@\protect\texttt{#1}}
    \pgfmanualbody
}
{
  \end{pgfmanualentry}
}


\def\pgfmanualbar{\char`\|}
\makeatletter
\newenvironment{pathoperation}[3][]{
  \begin{pgfmanualentry}
    \def\pgfmanualtest{#1}%
    \pgfmanualentryheadline{%
      \ifx\pgfmanualtest\@empty%
        \pgfmanualpdflabel{#2}{}%
      \fi%
      \textcolor{gray}{{\ttfamily\char`\\path}\
        \ \dots}
      \declare{\texttt{\noligs{#2}}}#3\ \textcolor{gray}{\dots\texttt{;}}}%
    \ifx\pgfmanualtest\@empty%
      \index{#2@\protect\texttt{#2} path operation}%
      \index{Path operations!#2@\protect\texttt{#2}}%
    \fi%
    \pgfmanualbody
}
{
  \end{pgfmanualentry}
}
\newenvironment{datavisualizationoperation}[3][]{
  \begin{pgfmanualentry}
    \def\pgfmanualtest{#1}%
    \pgfmanualentryheadline{%
      \ifx\pgfmanualtest\@empty%
        \pgfmanualpdflabel{#2}{}%
      \fi%
      \textcolor{gray}{{\ttfamily\char`\\datavisualization}\
        \ \dots}
      \declare{\texttt{\noligs{#2}}}#3\ \textcolor{gray}{\dots\texttt{;}}}%
    \ifx\pgfmanualtest\@empty%
      \index{#2@\protect\texttt{#2} (data visualization)}%
      \index{Data visualization!#2@\protect\texttt{#2}}%
    \fi%
    \pgfmanualbody
}
{
  \end{pgfmanualentry}
}
\makeatother

\def\doublebs{\texttt{\char`\\\char`\\}}


\newenvironment{package}[1]{
  \begin{pgfmanualentry}
    \pgfmanualentryheadline{%
      \pgfmanualpdflabel{#1}{}%
      {\ttfamily\char`\\usepackage\char`\{\declare{#1}\char`\}\space\space \char`\%\space\space  \LaTeX}}
    \index{#1@\protect\texttt{#1} package}%
    \index{Packages and files!#1@\protect\texttt{#1}}%
    \pgfmanualentryheadline{{\ttfamily\char`\\input \declare{#1}.tex\space\space\space \char`\%\space\space  plain \TeX}}
    \pgfmanualentryheadline{{\ttfamily\char`\\usemodule[\declare{#1}]\space\space \char`\%\space\space  Con\TeX t}}
    \pgfmanualbody
}
{
  \end{pgfmanualentry}
}


\newenvironment{pgfmodule}[1]{
  \begin{pgfmanualentry}
    \pgfmanualentryheadline{%
      \pgfmanualpdflabel{#1}{}%
      {\ttfamily\char`\\usepgfmodule\char`\{\declare{#1}\char`\}\space\space\space
        \char`\%\space\space  \LaTeX\space and plain \TeX\space and pure pgf}}
    \index{#1@\protect\texttt{#1} module}%
    \index{Modules!#1@\protect\texttt{#1}}%
    \pgfmanualentryheadline{{\ttfamily\char`\\usepgfmodule[\declare{#1}]\space\space \char`\%\space\space  Con\TeX t\space and pure pgf}}
    \pgfmanualbody
}
{
  \end{pgfmanualentry}
}

\newenvironment{pgflibrary}[1]{
  \begin{pgfmanualentry}
    \pgfmanualentryheadline{%
      \pgfmanualpdflabel{#1}{}%
      \textbf{\tikzname\ Library} \texttt{\declare{#1}}}
    \index{#1@\protect\texttt{#1} library}%
    \index{Libraries!#1@\protect\texttt{#1}}%
    \vskip.25em%
    {{\ttfamily\char`\\usepgflibrary\char`\{\declare{#1}\char`\}\space\space\space
        \char`\%\space\space  \LaTeX\space and plain \TeX\space and pure pgf}}\\
    {{\ttfamily\char`\\usepgflibrary[\declare{#1}]\space\space \char`\%\space\space  Con\TeX t\space and pure pgf}}\\
    {{\ttfamily\char`\\usetikzlibrary\char`\{\declare{#1}\char`\}\space\space
        \char`\%\space\space  \LaTeX\space and plain \TeX\space when using \tikzname}}\\
    {{\ttfamily\char`\\usetikzlibrary[\declare{#1}]\space
        \char`\%\space\space  Con\TeX t\space when using \tikzname}}\\[.5em]
    \pgfmanualbody
}
{
  \end{pgfmanualentry}
}

\newenvironment{purepgflibrary}[1]{
  \begin{pgfmanualentry}
    \pgfmanualentryheadline{%
      \pgfmanualpdflabel{#1}{}%
      \textbf{{\small PGF} Library} \texttt{\declare{#1}}}
    \index{#1@\protect\texttt{#1} library}%
    \index{Libraries!#1@\protect\texttt{#1}}%
    \vskip.25em%
    {{\ttfamily\char`\\usepgflibrary\char`\{\declare{#1}\char`\}\space\space\space
        \char`\%\space\space  \LaTeX\space and plain \TeX}}\\
    {{\ttfamily\char`\\usepgflibrary[\declare{#1}]\space\space \char`\%\space\space  Con\TeX t}}\\[.5em]
    \pgfmanualbody
}
{
  \end{pgfmanualentry}
}

\newenvironment{tikzlibrary}[1]{
  \begin{pgfmanualentry}
    \pgfmanualentryheadline{%
      \pgfmanualpdflabel{#1}{}%
      \textbf{\tikzname\ Library} \texttt{\declare{#1}}}
    \index{#1@\protect\texttt{#1} library}%
    \index{Libraries!#1@\protect\texttt{#1}}%
    \vskip.25em%
    {{\ttfamily\char`\\usetikzlibrary\char`\{\declare{#1}\char`\}\space\space \char`\%\space\space  \LaTeX\space and plain \TeX}}\\
    {{\ttfamily\char`\\usetikzlibrary[\declare{#1}]\space \char`\%\space\space Con\TeX t}}\\[.5em]
    \pgfmanualbody
}
{
  \end{pgfmanualentry}
}



\newenvironment{filedescription}[1]{
  \begin{pgfmanualentry}
    \pgfmanualentryheadline{File {\ttfamily\declare{#1}}}%
    \index{#1@\protect\texttt{#1} file}%
    \index{Packages and files!#1@\protect\texttt{#1}}%
    \pgfmanualbody
}
{
  \end{pgfmanualentry}
}


\newenvironment{packageoption}[1]{
  \begin{pgfmanualentry}
    \pgfmanualentryheadline{{\ttfamily\char`\\usepackage[\declare{#1}]\char`\{pgf\char`\}}}
    \index{#1@\protect\texttt{#1} package option}%
    \index{Package options for \textsc{pgf}!#1@\protect\texttt{#1}}%
    \pgfmanualbody
}
{
  \end{pgfmanualentry}
}



\newcommand\opt[1]{{\color{black!50!green}#1}}
\newcommand\ooarg[1]{{\ttfamily[}\meta{#1}{\ttfamily]}}

\def\opt{\afterassignment\pgfmanualopt\let\next=}
\def\pgfmanualopt{\ifx\next\bgroup\bgroup\color{black!50!green}\else{\color{black!50!green}\next}\fi}



\def\beamer{\textsc{beamer}}
\def\pdf{\textsc{pdf}}
\def\eps{\texttt{eps}}
\def\pgfname{\textsc{pgf}}
\def\tikzname{Ti\emph{k}Z}
\def\pstricks{\textsc{pstricks}}
\def\prosper{\textsc{prosper}}
\def\seminar{\textsc{seminar}}
\def\texpower{\textsc{texpower}}
\def\foils{\textsc{foils}}

{
  \makeatletter
  \global\let\myempty=\@empty
  \global\let\mygobble=\@gobble
  \catcode`\@=12
  \gdef\getridofats#1@#2\relax{%
    \def\getridtest{#2}%
    \ifx\getridtest\myempty%
      \expandafter\def\expandafter\strippedat\expandafter{\strippedat#1}
    \else%
      \expandafter\def\expandafter\strippedat\expandafter{\strippedat#1\protect\printanat}
      \getridofats#2\relax%
    \fi%
  }

  \gdef\removeats#1{%
    \let\strippedat\myempty%
    \edef\strippedtext{\stripcommand#1}%
    \expandafter\getridofats\strippedtext @\relax%
  }
  
  \gdef\stripcommand#1{\expandafter\mygobble\string#1}
}

\def\printanat{\char`\@}

\def\declare{\afterassignment\pgfmanualdeclare\let\next=}
\def\pgfmanualdeclare{\ifx\next\bgroup\bgroup\color{red!75!black}\else{\color{red!75!black}\next}\fi}


\let\textoken=\command
\let\endtextoken=\endcommand

\def\myprintocmmand#1{\texttt{\char`\\#1}}

\def\example{\par\smallskip\noindent\textit{Example: }}
\def\themeauthor{\par\smallskip\noindent\textit{Theme author: }}


\def\indexoption#1{%
  \index{#1@\protect\texttt{#1} option}%
  \index{Graphic options and styles!#1@\protect\texttt{#1}}%
}

\def\itemcalendaroption#1{\item \declare{\texttt{#1}}%
  \index{#1@\protect\texttt{#1} date test}%
  \index{Date tests!#1@\protect\texttt{#1}}%
}



\def\class#1{\list{}{\leftmargin=2em\itemindent-\leftmargin\def\makelabel##1{\hss##1}}%
\extractclass#1@\par\topsep=0pt}
\def\endclass{\endlist}
\def\extractclass#1#2@{%
\item{{{\ttfamily\char`\\documentclass}#2{\ttfamily\char`\{\declare{#1}\char`\}}}}%
  \index{#1@\protect\texttt{#1} class}%
  \index{Classes!#1@\protect\texttt{#1}}}

\def\partname{Part}

\makeatletter
\def\index@prologue{\section*{Index}\addcontentsline{toc}{section}{Index}
  This index only contains automatically generated entries. A good
  index should also contain carefully selected keywords. This index is
  not a good index.
  \bigskip
}
\c@IndexColumns=2
  \def\theindex{\@restonecoltrue
    \columnseprule \z@  \columnsep 29\p@
    \twocolumn[\index@prologue]%
       \parindent -30pt
       \columnsep 15pt
       \parskip 0pt plus 1pt
       \leftskip 30pt
       \rightskip 0pt plus 2cm
       \small
       \def\@idxitem{\par}%
    \let\item\@idxitem \ignorespaces}
  \def\endtheindex{\onecolumn}
\def\noindexing{\let\index=\@gobble}


\newenvironment{arrowtipsimple}[1]{
  \begin{pgfmanualentry}
    \pgfmanualentryheadline{\textbf{Arrow Tip Kind} {\ttfamily#1}}
    \index{#1@\protect\texttt{#1} arrow tip}%
    \index{Arrow tips!#1@\protect\texttt{#1}}%
    \def\currentarrowtype{#1}
    \pgfmanualbody}
{
  \end{pgfmanualentry}
}

\newenvironment{arrowtip}[4]{
  \begin{pgfmanualentry}
    \pgfmanualentryheadline{\textbf{Arrow Tip Kind} {\ttfamily#1}}
    \index{#1@\protect\texttt{#1} arrow tip}%
    \index{Arrow tips!#1@\protect\texttt{#1}}%
    \pgfmanualbody
    \def\currentarrowtype{#1}
    \begin{minipage}[t]{10.25cm}
      #2
    \end{minipage}\hskip5mm\begin{minipage}[t]{4.75cm}
      \leavevmode\vskip-2em
    \tikz{
      \draw [black!50,line width=5mm,-{#1[#3,color=black]}] (-4,0) -- (0,0);
      \foreach \action in {#4}
      { \expandafter\processaction\action\relax }
    }
    \end{minipage}\par\smallskip
  }
{
  \end{pgfmanualentry}
}

\newenvironment{arrowcap}[5]{
  \begin{pgfmanualentry}
    \pgfmanualentryheadline{\textbf{Arrow Tip Kind} {\ttfamily#1}}
    \index{#1@\protect\texttt{#1} arrow tip}%
    \index{Arrow tips!#1@\protect\texttt{#1}}%
    \pgfmanualbody
    \def\currentarrowtype{#1}
    \begin{minipage}[t]{10.25cm}
      #2
    \end{minipage}\hskip5mm\begin{minipage}[t]{4.75cm}
      \leavevmode\vskip-2em
    \tikz{
      \path [tips, line width=10mm,-{#1[#3,color=black]}] (-4,0) -- (0,0);
      \draw [line width=10mm,black!50] (-3,0) -- (#5,0);
      \foreach \action in {#4}
      { \expandafter\processaction\action\relax }
    }
    \end{minipage}\par\smallskip
  }
{
  \end{pgfmanualentry}
}

\newenvironment{pattern}[1]{
  \begin{pgfmanualentry}
    \pgfmanualentryheadline{\textbf{Pattern} {\ttfamily#1}}
    \index{#1@\protect\texttt{#1} pattern}%
    \index{Patterns!#1@\protect\texttt{#1}}%
    \pgfmanualbody
}
{
  \end{pgfmanualentry}
}

\def\processaction#1=#2\relax{
  \expandafter\let\expandafter\pgf@temp\csname manual@action@#1\endcsname
  \ifx\pgf@temp\relax\else
    \pgf@temp#2/0/\relax
  \fi
}
\def\manual@action@length#1/#2/#3\relax{%
  \draw [red,|<->|,semithick,xshift=#2] ([yshift=4pt]current bounding
  box.north -| -#1,0) coordinate (last length) -- node
  [above=-2pt] {|length|} ++(#1,0);
}
\def\manual@action@width#1/#2/#3\relax{%
  \draw [overlay, red,|<->|,semithick] (.5,-#1/2) -- node [below,sloped] {|width|} (.5,#1/2);
}
\def\manual@action@inset#1/#2/#3\relax{%
  \draw [red,|<->|,semithick,xshift=#2] ([yshift=-4pt]current bounding
  box.south -| last length) -- node [below] {|inset|} ++(#1,0);
}

\newenvironment{arrowexamples}
{\begin{tabbing}
    \hbox to \dimexpr\linewidth-5.5cm\relax{\emph{Appearance of the below at line width} \hfil} \= 
     \hbox to 1.9cm{\emph{0.4pt}\hfil} \= \hbox to 2cm{\emph{0.8pt}\hfil} \= \emph{1.6pt} \\
  }
{\end{tabbing}\vskip-1em}

\newenvironment{arrowcapexamples}
{\begin{tabbing}
    \hbox to \dimexpr\linewidth-5.5cm\relax{\emph{Appearance of the below at line width} \hfil} \= 
     \hbox to 1.9cm{\emph{1ex}\hfil} \= \hbox to 2cm{\emph{1em}\hfil} \\
  }
{\end{tabbing}\vskip-1em}

\def\arrowcapexample#1[#2]{\def\temp{#1}\ifx\temp\pgfutil@empty\arrowcapexample@\currentarrowtype[{#2}]\else\arrowcapexample@#1[{#2}]\fi}
\def\arrowcapexample@#1[#2]{%
  {\sfcode`\.1000\small\texttt{#1[#2]}} \>
  \kern-.5ex\tikz [baseline,>={#1[#2]}] \draw [line
  width=1ex,->] (0,.5ex) -- (2em,.5ex);  \>
  \kern-.5em\tikz [baseline,>={#1[#2]}] \draw [line
  width=1em,->] (0,.5ex) -- (2em,.5ex);  \\
}

\def\arrowexample#1[#2]{\def\temp{#1}\ifx\temp\pgfutil@empty\arrowexample@\currentarrowtype[{#2}]\else\arrowexample@#1[{#2}]\fi}
\def\arrowexample@#1[#2]{%
  {\sfcode`\.1000\small\texttt{#1[#2]}} \>
  \tikz [baseline,>={#1[#2]}] \draw [line
  width=0.4pt,->] (0,.5ex) -- (2em,.5ex); thin \>
  \tikz [baseline,>={#1[#2]}] \draw [line
  width=0.8pt,->] (0,.5ex) -- (2em,.5ex); \textbf{thick} \>
  \tikz [baseline,>={#1[#2]}] \draw [line
  width=1.6pt,->] (0,.5ex) -- (3em,.5ex); \\
}
\def\arrowexampledup[#1]{\arrowexample[{#1] \currentarrowtype[}]}
\def\arrowexampledupdot[#1]{\arrowexample[{#1] . \currentarrowtype[}]}

\def\arrowexampledouble#1[#2]{\def\temp{#1}\ifx\temp\pgfutil@empty\arrowexampledouble@\currentarrowtype[{#2}]\else\arrowexampledouble@#1[{#2}]\fi}
\def\arrowexampledouble@#1[#2]{%
  {\sfcode`\.1000\small\texttt{#1[#2]} on double line} \>
  \tikz [baseline,>={#1[#2]}]
    \draw [double equal sign distance,line width=0.4pt,->] (0,.5ex) -- (2em,.5ex); thin \>
  \tikz [baseline,>={#1[#2]}]
    \draw [double equal sign distance,line width=0.8pt,->] (0,.5ex) -- (2em,.5ex); \textbf{thick} \>
  \tikz [baseline,>={#1[#2]}]
    \draw [double equal sign distance, line width=1.6pt,->] (0,.5ex) -- (3em,.5ex); \\
}



\newcommand\symarrow[1]{%
  \index{#1@\protect\texttt{#1} arrow tip}%
  \index{Arrow tips!#1@\protect\texttt{#1}}%
  \texttt{#1}& yields thick  
  \begin{tikzpicture}[arrows={#1-#1},thick,baseline]
    \useasboundingbox (-1mm,-0.5ex) rectangle (1.1cm,2ex);
    \fill [black!15] (1cm,-.5ex) rectangle (1.1cm,1.5ex) (-1mm,-.5ex) rectangle (0mm,1.5ex) ;
    \draw (0pt,.5ex) -- (1cm,.5ex);
  \end{tikzpicture} and thin
  \begin{tikzpicture}[arrows={#1-#1},thin,baseline]
    \useasboundingbox (-1mm,-0.5ex) rectangle (1.1cm,2ex);
    \fill [black!15] (1cm,-.5ex) rectangle (1.1cm,1.5ex) (-1mm,-.5ex) rectangle (0mm,1.5ex) ;
    \draw (0pt,.5ex) -- (1cm,.5ex);
  \end{tikzpicture}
}
\newcommand\symarrowdouble[1]{%
  \index{#1@\protect\texttt{#1} arrow tip}%
  \index{Arrow tips!#1@\protect\texttt{#1}}%
  \texttt{#1}& yields thick  
  \begin{tikzpicture}[arrows={#1-#1},thick,baseline]
    \useasboundingbox (-1mm,-0.5ex) rectangle (1.1cm,2ex);
    \fill [black!15] (1cm,-.5ex) rectangle (1.1cm,1.5ex) (-1mm,-.5ex) rectangle (0mm,1.5ex) ;
    \draw (0pt,.5ex) -- (1cm,.5ex);
  \end{tikzpicture}
  and thin
  \begin{tikzpicture}[arrows={#1-#1},thin,baseline]
    \useasboundingbox (-1mm,-0.5ex) rectangle (1.1cm,2ex);
    \fill [black!15] (1cm,-.5ex) rectangle (1.1cm,1.5ex) (-1mm,-.5ex) rectangle (0mm,1.5ex) ;
    \draw (0pt,.5ex) -- (1cm,.5ex);
  \end{tikzpicture}, double 
  \begin{tikzpicture}[arrows={#1-#1},thick,baseline]
    \useasboundingbox (-1mm,-0.5ex) rectangle (1.1cm,2ex);
    \fill [black!15] (1cm,-.5ex) rectangle (1.1cm,1.5ex) (-1mm,-.5ex) rectangle (0mm,1.5ex) ;
    \draw[double,double equal sign distance] (0pt,.5ex) -- (1cm,.5ex);
  \end{tikzpicture} and 
  \begin{tikzpicture}[arrows={#1-#1},thin,baseline]
    \useasboundingbox (-1mm,-0.5ex) rectangle (1.1cm,2ex);
    \fill [black!15] (1cm,-.5ex) rectangle (1.1cm,1.5ex) (-1mm,-.5ex) rectangle (0mm,1.5ex) ;
    \draw[double,double equal sign distance] (0pt,.5ex) -- (1cm,.5ex);
  \end{tikzpicture}
}

\newcommand\sarrow[2]{%
  \index{#1@\protect\texttt{#1} arrow tip}%
  \index{Arrow tips!#1@\protect\texttt{#1}}%
  \index{#2@\protect\texttt{#2} arrow tip}%
  \index{Arrow tips!#2@\protect\texttt{#2}}%
  \texttt{#1-#2}& yields thick  
  \begin{tikzpicture}[arrows={#1-#2},thick,baseline]
    \useasboundingbox (-1mm,-0.5ex) rectangle (1.1cm,2ex);
    \fill [black!15] (1cm,-.5ex) rectangle (1.1cm,1.5ex) (-1mm,-.5ex) rectangle (0mm,1.5ex) ;
    \draw (0pt,.5ex) -- (1cm,.5ex);
  \end{tikzpicture} and thin
  \begin{tikzpicture}[arrows={#1-#2},thin,baseline]
    \useasboundingbox (-1mm,-0.5ex) rectangle (1.1cm,2ex);
    \fill [black!15] (1cm,-.5ex) rectangle (1.1cm,1.5ex) (-1mm,-.5ex) rectangle (0mm,1.5ex) ;
    \draw (0pt,.5ex) -- (1cm,.5ex);
  \end{tikzpicture}
}

\newcommand\sarrowdouble[2]{%
  \index{#1@\protect\texttt{#1} arrow tip}%
  \index{Arrow tips!#1@\protect\texttt{#1}}%
  \index{#2@\protect\texttt{#2} arrow tip}%
  \index{Arrow tips!#2@\protect\texttt{#2}}%
  \texttt{#1-#2}& yields thick  
  \begin{tikzpicture}[arrows={#1-#2},thick,baseline]
    \useasboundingbox (-1mm,-0.5ex) rectangle (1.1cm,2ex);
    \fill [black!15] (1cm,-.5ex) rectangle (1.1cm,1.5ex) (-1mm,-.5ex) rectangle (0mm,1.5ex) ;
    \draw (0pt,.5ex) -- (1cm,.5ex);
  \end{tikzpicture} and thin
  \begin{tikzpicture}[arrows={#1-#2},thin,baseline]
    \useasboundingbox (-1mm,-0.5ex) rectangle (1.1cm,2ex);
    \fill [black!15] (1cm,-.5ex) rectangle (1.1cm,1.5ex) (-1mm,-.5ex) rectangle (0mm,1.5ex) ;
    \draw (0pt,.5ex) -- (1cm,.5ex);
  \end{tikzpicture}, double 
  \begin{tikzpicture}[arrows={#1-#2},thick,baseline]
    \useasboundingbox (-1mm,-0.5ex) rectangle (1.1cm,2ex);
    \fill [black!15] (1cm,-.5ex) rectangle (1.1cm,1.5ex) (-1mm,-.5ex) rectangle (0mm,1.5ex) ;
    \draw[double,double equal sign distance] (0pt,.5ex) -- (1cm,.5ex);
  \end{tikzpicture} and 
  \begin{tikzpicture}[arrows={#1-#2},thin,baseline]
    \useasboundingbox (-1mm,-0.5ex) rectangle (1.1cm,2ex);
    \fill [black!15] (1cm,-.5ex) rectangle (1.1cm,1.5ex) (-1mm,-.5ex) rectangle (0mm,1.5ex) ;
    \draw[double,double equal sign distance] (0pt,.5ex) -- (1cm,.5ex);
  \end{tikzpicture}
}

\newcommand\carrow[1]{%
  \index{#1@\protect\texttt{#1} arrow tip}%
  \index{Arrow tips!#1@\protect\texttt{#1}}%
  \texttt{#1}& yields for line width 1ex
  \begin{tikzpicture}[arrows={#1-#1},line width=1ex,baseline]
    \useasboundingbox (-1mm,-0.5ex) rectangle (1.6cm,2ex);
    \fill [black!15] (1.5cm,-.5ex) rectangle (1.6cm,1.5ex) (-1mm,-.5ex) rectangle (0mm,1.5ex) ;
    \draw (0pt,.5ex) -- (1.5cm,.5ex);
  \end{tikzpicture}
}
\def\myvbar{\char`\|}
\newcommand\plotmarkentry[1]{%
  \index{#1@\protect\texttt{#1} plot mark}%
  \index{Plot marks!#1@\protect\texttt{#1}}
  \texttt{\char`\\pgfuseplotmark\char`\{\declare{\noligs{#1}}\char`\}} &
  \tikz\draw[color=black!25] plot[mark=#1,mark options={fill=examplefill,draw=black}] coordinates{(0,0) (.5,0.2) (1,0) (1.5,0.2)};\\
}
\newcommand\plotmarkentrytikz[1]{%
  \index{#1@\protect\texttt{#1} plot mark}%
  \index{Plot marks!#1@\protect\texttt{#1}}
  \texttt{mark=\declare{\noligs{#1}}} & \tikz\draw[color=black!25]
  plot[mark=#1,mark options={fill=examplefill,draw=black}] 
    coordinates {(0,0) (.5,0.2) (1,0) (1.5,0.2)};\\
}



\ifx\scantokens\@undefined
  \PackageError{pgfmanual-macros}{You need to use extended latex
    (elatex) or (pdfelatex) to process this document}{}
\fi

\begingroup
\catcode`|=0
\catcode`[= 1
\catcode`]=2
\catcode`\{=12
\catcode `\}=12
\catcode`\\=12 |gdef|find@example#1\end{codeexample}[|endofcodeexample[#1]]
|endgroup

% define \returntospace.
%
% It should define NEWLINE as {}, spaces and tabs as \space.
\begingroup
\catcode`\^=7
\catcode`\^^M=13
\catcode`\^^I=13
\catcode`\ =13%
\gdef\returntospace{\catcode`\ =13\def {\space}\catcode`\^^I=13\def^^I{\space}}
\gdef\showreturn{\show^^M}
\endgroup

\begingroup
\catcode`\%=13
\catcode`\^^M=13
\gdef\commenthandler{\catcode`\%=13\def%{\@gobble@till@return}}
\gdef\@gobble@till@return#1^^M{}
\gdef\@gobble@till@return@ignore#1^^M{\ignorespaces}
\gdef\typesetcomment{\catcode`\%=13\def%{\@typeset@till@return}}
\gdef\@typeset@till@return#1^^M{{\def%{\char`\%}\textsl{\char`\%#1}}\par}
\endgroup

% Define tab-implementation functions
%   \codeexample@tabinit@replacementchars@
% and
%   \codeexample@tabinit@catcode@
%
% They should ONLY be used in case that tab replacement is active.
%
% This here is merely a preparation step.
%
% Idea:
% \codeexample@tabinit@catcode@ will make TAB active
% and
% \codeexample@tabinit@replacementchars@ will insert as many spaces as
% /codeexample/tabsize contains.
{
\catcode`\^^I=13
% ATTENTION: do NOT use tabs in these definitions!!
\gdef\codeexample@tabinit@replacementchars@{%
 \begingroup
 \count0=\pgfkeysvalueof{/codeexample/tabsize}\relax
 \toks0={}%
 \loop
 \ifnum\count0>0
  \advance\count0 by-1
  \toks0=\expandafter{\the\toks0\ }%
 \repeat
 \xdef\codeexample@tabinit@replacementchars@@{\the\toks0}%
 \endgroup
 \let^^I=\codeexample@tabinit@replacementchars@@
}%
\gdef\codeexample@tabinit@catcode@{\catcode`\^^I=13}%
}%

% Called after any options have been set. It assigns
%   \codeexample@tabinit@catcode
% and
%   \codeexample@tabinit@replacementchars
% which are used inside of 
%\begin{codeexample}
% ...
%\end{codeexample}
%
% \codeexample@tabinit@catcode  is either \relax or it makes tab
% active.
%
% \codeexample@tabinit@replacementchars is either \relax or it inserts
% a proper replacement sequence for tabs (as many spaces as
% configured)
\def\codeexample@tabinit{%
  \ifnum\pgfkeysvalueof{/codeexample/tabsize}=0\relax
    \let\codeexample@tabinit@replacementchars=\relax
    \let\codeexample@tabinit@catcode=\relax
  \else
    \let\codeexample@tabinit@catcode=\codeexample@tabinit@catcode@
    \let\codeexample@tabinit@replacementchars=\codeexample@tabinit@replacementchars@
  \fi
}

\newif\ifpgfmanualtikzsyntaxhilighting

\pgfqkeys{/codeexample}{%
  width/.code=  {\setlength\codeexamplewidth{#1}},
  graphic/.code=  {\colorlet{graphicbackground}{#1}},
  code/.code=  {\colorlet{codebackground}{#1}},
  execute code/.is if=code@execute,
  hidden/.is if=code@hidden,
  code only/.code=  {\code@executefalse},
  setup code/.code=  {\pgfmanual@setup@codetrue\code@executefalse},
  multipage/.code=  {\code@executefalse\pgfmanual@multipage@codetrue},
  pre/.store in=\code@pre,
  post/.store in=\code@post,
  % #1 is the *complete* environment contents as it shall be
  % typeset. In particular, the catcodes are NOT the normal ones.
  typeset listing/.code=  {#1},
  render instead/.store in=\code@render,
  vbox/.code=  {\def\code@pre{\vbox\bgroup\setlength{\hsize}{\linewidth-6pt}}\def\code@post{\egroup}},
  ignorespaces/.code=  {\let\@gobble@till@return=\@gobble@till@return@ignore},
  leave comments/.code=  {\def\code@catcode@hook{\catcode`\%=12}\let\commenthandler=\relax\let\typesetcomment=\relax},
  tabsize/.initial=0,% FIXME : this here is merely used for indentation. It is just a TAB REPLACEMENT.
  every codeexample/.style={width=4cm+7pt, tikz syntax=true},
  from file/.code={\codeexamplefromfiletrue\def\codeexamplesource{#1}},
  tikz syntax/.is if=pgfmanualtikzsyntaxhilighting,
  animation list/.store in=\code@animation@list,
  animation pre/.store in=\code@animation@pre,
  animation post/.store in=\code@animation@post,
  animation scale/.store in=\pgfmanualanimscale,
  animation bb/.style={
    animation pre={
      \tikzset{
        every picture/.style={
          execute at begin picture={
            \useasboundingbox[clip] #1;}
        }
      }
    }
  },
  preamble/.store in=\code@preamble,
}

\def\pgfmanualanimscale{.5}

\newread\examplesource


% Opening, reading and closing the results file

\def\opensource#1{
  \immediate\openin\examplesource=#1
}
\def\do@codeexamplefromfile{%
  \immediate\openin\examplesource\expandafter{\codeexamplesource}%
  \def\examplelines{}%
  \readexamplelines
  \closein\examplesource
  \expandafter\endofcodeexample\expandafter{\examplelines}%
}

\def\readexamplelines{
  \ifeof\examplesource%
  \else
    \immediate\read\examplesource to \exampleline
    \expandafter\expandafter\expandafter\def\expandafter\expandafter\expandafter\examplelines\expandafter\expandafter\expandafter{\expandafter\examplelines\exampleline}
    \expandafter\readexamplelines%
  \fi
}

\let\code@animation@pre\pgfutil@empty
\let\code@animation@post\pgfutil@empty
\let\code@animation@list\pgfutil@empty

\let\code@pre\pgfutil@empty
\let\code@post\pgfutil@empty
\let\code@render\pgfutil@empty
\let\code@preamble\pgfutil@empty
\def\code@catcode@hook{}

\newif\ifpgfmanual@multipage@code
\newif\ifpgfmanual@setup@code
\newif\ifcodeexamplefromfile
\newdimen\codeexamplewidth
\newif\ifcode@execute
\newif\ifcode@hidden
\newbox\codeexamplebox
\def\codeexample[#1]{%
  \global\let\pgfmanual@do@this\relax%
  \aftergroup\pgfmanual@do@this%
  \begingroup%
  \code@executetrue
  \pgfqkeys{/codeexample}{every codeexample,#1}%
  \pgfmanualswitchoncolors%
  \ifcodeexamplefromfile\begingroup\fi
  \codeexample@tabinit% assigns \codeexample@tabinit@[catcode,replacementchars]
  \parindent0pt
  \begingroup%
  \par% this \par is not inside \ifcode@hidden because we want to switch to vmode
  \ifcode@hidden\else
    \medskip%
  \fi
  \let\do\@makeother%
  \dospecials%
  \obeylines%
  \@vobeyspaces%
  \catcode`\%=13%
  \catcode`\^^M=13%
  \code@catcode@hook%
  \codeexample@tabinit@catcode
  \relax%
  \ifcodeexamplefromfile%
    \expandafter\do@codeexamplefromfile%
  \else%
    \expandafter\find@example%
  \fi}
\def\endofcodeexample#1{%
  \endgroup%
  \ifpgfmanual@setup@code%
    \gdef\pgfmanual@do@this{%
      {%
        \returntospace%
        \commenthandler%
        \xdef\code@temp{#1}% removes returns and comments
      }%
      \edef\pgfmanualmcatcode{\the\catcode`\^^M}%
      \catcode`\^^M=9\relax%
      \expandafter\scantokens\expandafter{\code@temp}%
      \catcode`\^^M=\pgfmanualmcatcode%
    }%
  \fi%
  \ifcode@hidden\else
    \ifcode@execute%
      \setbox\codeexamplebox=\hbox{%
        \ifx\code@render\pgfutil@empty%
        {%
          {%
            \returntospace%
            \commenthandler%
            \xdef\code@temp{#1}% removes returns and comments
          }%
          \catcode`\^^M=9%
          \colorbox{graphicbackground}{\color{black}\ignorespaces%
            \code@pre\expandafter\scantokens\expandafter{\code@temp\ignorespaces}\code@post\ignorespaces}%
        }%
        \else%
          \global\let\code@temp\code@render%
          \colorbox{graphicbackground}{\color{black}\ignorespaces%
            \code@render}%
        \fi%
      }%
      \ifx\code@animation@list\pgfutil@empty%
      \else%
      \setbox\codeexampleboxanim=\vbox{%
        \rightskip0pt\leftskip0pt plus1filll%
        \ifdim\wd\codeexamplebox>\codeexamplewidth%
        \else%
          \hsize\codeexamplewidth%
          \advance\hsize by2cm%
        \fi%
        \leavevmode\catcode`\^^M=9%
        \foreach \pgfmanualtime/\pgfmanualtimehow in\code@animation@list{%
          \setbox\codeexampleboxanim=\hbox{\colorbox{animationgraphicbackground}{%
              \tikzset{make snapshot of=\pgfmanualtime}%
              \scalebox{\pgfmanualanimscale}{\color{black}\ignorespaces%
                \code@animation@pre\expandafter\scantokens\expandafter{\code@temp\ignorespaces}\code@animation@post\ignorespaces}%
            }}%
          \space\raise4pt\hbox to0pt{\vrule width0pt height1em\hbox
            to\wd\codeexampleboxanim{\hfil\scriptsize$t{=}\pgfmanualtimehow \mathrm s$\hfil}\hss}%
          \lower\ht\codeexampleboxanim\box\codeexampleboxanim\hfil\penalty0\hskip0ptplus-1fil%
        }%
      }%
      \setbox\codeexampleboxanim=\hbox{\hbox{}\hskip-2cm\box\codeexampleboxanim}%
      \fi%
      \ifdim\wd\codeexamplebox>\codeexamplewidth%
        \def\code@start{\par}%
        \def\code@flushstart{}\def\code@flushend{}%
        \def\code@mid{\parskip2pt\par\noindent}%
        \def\code@width{\linewidth-6pt}%
        \def\code@end{}%
      \else%
        \def\code@start{%
          \linewidth=\textwidth%
          \parshape \@ne 0pt \linewidth
          \leavevmode%
          \hbox\bgroup}%
        \def\code@flushstart{\hfill}%
        \def\code@flushend{\hbox{}}%
        \def\code@mid{\hskip6pt}%
        \def\code@width{\linewidth-12pt-\codeexamplewidth}%
        \def\code@end{\egroup}%
      \fi%
      \code@start%
      \noindent%
      \begin{minipage}[t]{\codeexamplewidth}\raggedright
        \hrule width0pt%
        \footnotesize\vskip-1em%
        \code@flushstart\box\codeexamplebox\code@flushend%
        \vskip0pt%
        \leavevmode%
        \box\codeexampleboxanim%
        \vskip-1ex
        \leavevmode%
      \end{minipage}%
    \else%
      \def\code@mid{\par}
      \def\code@width{\linewidth-6pt}
      \def\code@end{}
    \fi%
    \code@mid%
      \ifpgfmanual@multipage@code%
        {%
          \pgfkeysvalueof{/codeexample/prettyprint/base color}%
          \pgfmanualdolisting{#1}%
        }%
      \else%
        \colorbox{codebackground}{%
          \pgfkeysvalueof{/codeexample/prettyprint/base color}%
          \begin{minipage}[t]{\code@width}%
            \pgfmanualdolisting{#1}%
          \end{minipage}}%
      \fi%
    \code@end%
    \par%
    \medskip
  \fi
  \endcodeexample\endgroup%
}

\def\endcodeexample{\endgroup}
\newbox\codeexampleboxanim

\def\pgfmanualdolisting#1{%
      {%
        \let\do\@makeother
        \dospecials
        \frenchspacing\@vobeyspaces
        \normalfont\ttfamily\footnotesize
        \typesetcomment%
        \codeexample@tabinit@replacementchars
        \@tempswafalse
        \def\par{%
          \if@tempswa
          \leavevmode \null \@@par\penalty\interlinepenalty
          \else
          \@tempswatrue
          \ifhmode\@@par\penalty\interlinepenalty\fi
          \fi}%
        \obeylines
        \everypar \expandafter{\the\everypar \unpenalty}%
        \ifx\code@preamble\pgfutil@empty\else
          \pgfutil@tempdima=\hsize
          \vbox{\hsize=\pgfutil@tempdima
              \raggedright\scriptsize\detokenize\expandafter{\code@preamble}}%
        \fi
        \pgfkeysvalueof{/codeexample/typeset listing/.@cmd}{#1}\pgfeov
      }%
}

\makeatother

\usepackage{pgfmanual}


% autoxref is now always on

% \makeatletter
% % \pgfautoxrefs will be defined by 'make dist'
% \pgfutil@ifundefined{pgfautoxrefs}{%
%   \renewcommand\pgfmanualpdflabel[3][]{#3}% NO-OP
%   \def\pgfmanualpdfref#1#2{#2}%
%   \pgfkeys{
%     /pdflinks/codeexample links=false,% DISABLED.
%   }%
% }{}
% \makeatother

\newdimen\pgfmanualcslinkpreskip

% Styling of the pretty printer
\pgfkeys{
  /codeexample/syntax hilighting/.style={
    /codeexample/prettyprint/key name/.code={\textcolor{keycolor}{\pgfmanualpdfref{##1}{\noligs{##1}}}},
    /codeexample/prettyprint/key name with handler/.code 2 args={\textcolor{keycolor}{\pgfmanualpdfref{##1}{\noligs{##1}}}/\textcolor{blue!70!black}{\pgfmanualpdfref{/handlers/##2}{\noligs{##2}}}},
    /codeexample/prettyprint/key value display only/.code={\textcolor{keycolor}{{\itshape{\let\pgfmanualwordstartup\relax\pgfmanualprettyprintcode{##1}}}}},
    /codeexample/prettyprint/cs/.code={\textcolor{cscolor}{\pgfmanualcslinkpreskip4.25pt\pgfmanualpdfref{##1}{\noligs{##1}}}},
    /codeexample/prettyprint/cs with args/.code 2 args={\textcolor{black}{\pgfmanualcslinkpreskip4.25pt\pgfmanualpdfref{##1}{\noligs{##1}}}\{\textcolor{black}{\pgfmanualprettyprintcode{##2}}\pgfmanualclosebrace},
    /codeexample/prettyprint/cs arguments/pgfkeys/.initial=1,
    /codeexample/prettyprint/cs/pgfkeys/.code 2 args={\textcolor{black}{\pgfmanualcslinkpreskip4.25pt\pgfmanualpdfref{##1}{\noligs{##1}}}\{\textcolor{black}{\pgfmanualprettyprintpgfkeys{##2}}\pgfmanualclosebrace},
    /codeexample/prettyprint/cs arguments/begin/.initial=1,
    /codeexample/prettyprint/cs/begin/.code 2 args={\textcolor{black}{##1}\{\textcolor{cscolor}{\pgfmanualpdfref{##2}{\noligs{##2}}}\pgfmanualclosebrace},
    /codeexample/prettyprint/cs arguments/end/.initial=1,
    /codeexample/prettyprint/cs/end/.code 2 args={\textcolor{black}{##1}\{\textcolor{cscolor}{\pgfmanualpdfref{##2}{\noligs{##2}}}\pgfmanualclosebrace},
    /codeexample/prettyprint/word/.code={\pgfmanualwordstartup{\begingroup\pgfkeyssetvalue{/pdflinks/search key prefixes in}{}\pgfmanualpdfref{##1}{\noligs{##1}}\endgroup}},
    /codeexample/prettyprint/point/.code={\textcolor{pointcolor}{\noligs{##1}}},%
    /codeexample/prettyprint/point with cs/.code 2 args={\textcolor{pointcolor}{(\pgfmanualpdfref{##1}{\noligs{##1}}:\noligs{##2}}},%
    /codeexample/prettyprint/comment font=\itshape,
    /codeexample/prettyprint/base color/.initial=\color{basecolor},
    /pdflinks/render hyperlink/.code={%
      {\setbox0=\hbox{##1}%
        \rlap{{\color{linkcolor}\dimen0\wd0\advance\dimen0by-\pgfmanualcslinkpreskip\hskip\pgfmanualcslinkpreskip\vrule width\dimen0 height-1pt depth1.6pt}}%
        \box0%
      }%
    }
  },/codeexample/syntax hilighting
}

\colorlet{keycolor}{black}
\colorlet{cscolor}{black}
\colorlet{pointcolor}{black}
\colorlet{basecolor}{black}
\colorlet{linkcolor}{black!8}

\def\pgfmanualswitchoncolors{%
  \colorlet{keycolor}{green!50!black}%
  \colorlet{cscolor}{blue!70!black}
  \colorlet{pointcolor}{violet}
  \colorlet{basecolor}{black!55}
  \colorlet{linkcolor}{white}
}

\makeatletter

\def\pgfmanualwordstartup{\textcolor{black}}

\def\noligs#1{\pgfmanualnoligs#1\kern0pt--\pgf@stop}%
\def\pgfmanualnoligs#1--{%
  \pgfutil@ifnextchar\pgf@stop{#1\pgfutil@gobble}{#1-\kern0pt-\kern0pt\pgfmanualnoligs}%
}
\makeatother


%%% Local Variables: 
%%% mode: latex
%%% TeX-master: "beameruserguide"
%%% End: 
 % Is supposed to be included in recent TeX distributions, but I get errors...
\usepackage{makeidx} % Produces an index of commands.
\makeindex % Useful or not index will be created
\usepackage[hidelinks]{hyperref}
\newcommand{\mylink}[2]{\href{#1}{#2}\footnote{\url{#1}}}
\usepackage{verbatim}

%%%%%%%%%%%%%%%%%%%%%%%%%%%%%%
%%% Documentation
%%%%%%%%%%%%%%%%%%%%%%%%%%%%%%

\begin{document}
%%% Title: thanks tikzcd for the styling
\begin{center}
  \vspace*{1em} % Thanks tikzcd
  \tikz\node[scale=1.2]{%
    \color{gray}\Huge\ttfamily \char`\{\raisebox{.09em}{\textcolor{red!75!black}{zx}}\char`\}};

  \vspace{0.5em}
  {\Large\bfseries ZX-calculus with \tikzname}

  \vspace{1em}
  {Version 1.0 \qquad October 12, 2021}\\[2mm]
  {\href{https://github.com/leo-colisson/zx}{\texttt{github.com/leo-colisson/zx}}}
\end{center}

\tableofcontents

\section{Introduction}

This library (based on the great \tikzname{} and \tikzname-cd packages) allows you to typeset ZX-calculus directly in \LaTeX{}. It comes with a default---but highly customizable---style:
\begin{codeexample}[]
  \begin{ZX}
    \zxZ{\alpha} \arrow[r] & \zxFracX-{\pi}{4}
  \end{ZX}
\end{codeexample}
Even if this has not yet been tested a lot, you can also use a ``phase in label'' style, without really changing the code:
\begin{codeexample}[]
  \begin{ZX}[phase in label right]
    \zxZ{\alpha} \arrow[d] \\
    \zxFracX-{\pi}{4}
  \end{ZX}
\end{codeexample}

The goal is to provide an alternative to the great |tikzit| package: we wanted a solution that does not require the creation of an additional file, the use of an external software, and which automatically adapts the width of columns and rows depending on the content of the nodes (in |tikzit| one needs to manually tune the position of each node, especially when dealing with large nodes). Our library also provides a default style and tries to separate the content from the style: that way it should be easy to globally change the styling of a given project without redesigning all diagrams. However, it should be fairly easy to combine tikzit and this library: when some diagrams are easier to design in tikzit, then it should be possible to directly load the style of this library inside tikzit.

This library is quite young, so feel free to propose improvements or report issues on \href{https://github.com/leo-colisson/zx}{\texttt{github.com/leo-colisson/zx}}. We will of course try to maintain backward compatibility as much as possible, but we can't guarantee at 100\% that small changes (spacing\dots{}) won't be made later. In case you want a completely unalterable style, just copy this library in your project forever! The documentation is also a work in progress, so feel free to check the code to discover new functionalities.

\section{Installation}

If your CTAN distribution is recent enough, you can directly insert in your file:
% verse indents stuff, index adds to the index of command at the end of the file, || is a shortcut of \verb||
\begin{verse}
  \index{zx@\protect\texttt{zx} package}%
  \index{Packages and files!zx@\protect\texttt{tikz-cd}}%
  |\usepackage{zx}|%
\end{verse}
or load \tikzname{} and then use:
\begin{verse}%
   \index{cd@\protect\texttt{cd} library}%
   \index{Libraries!cd@\protect\texttt{cd}}%
   |\usetikzlibrary{cd}|%
\end{verse}
If this library is not yet packaged into CTAN (which is very likely in 2021), you must first download \mylink{https://github.com/leo-colisson/zx/blob/main/tikzlibraryzx.code.tex}{\texttt{tikzlibraryzx.code.tex}} and \mylink{https://github.com/leo-colisson/zx/blob/main/zx.sty}{\texttt{zx.sty}} (right-click on ``Raw'' and ``Save link as'') and save them at the root of your project.

\section{Usage}

\subsection{Add a diagram}
\begin{pgfmanualentry}
  \extractcommand\zx\opt{\oarg{options}}\marg{your diagram}\@@
  \extractenvironement{ZX}\opt{\oarg{options}}\@@
  \pgfmanualbody
  You can create a new ZX-diagram either with a command (quicker for inline diagrams) or with an environment. The \meta{options} can be used to locally change the style of the diagram, using the same options as the |{tikz-cd}| environment (from the \mylink{https://www.ctan.org/pkg/tikz-cd}{\texttt{tikz-cd} package}). The \meta{your diagram} argument, or the content of |{ZX}| environment is a \tikzname{} matrix of nodes, exactly like in the |tikz-cd| package: columns are separated using |&|, columns using |\\|, and nodes are created using \verb#|[tikz style]| node content# or with shortcut commands presented later in this document (recommended). Content is typeset in math mode by default, and diagrams can be included in any equation. Wires can be added like in |tikz-cd| (see more below) using |\arrow| or |\ar|: we provide later recommended styles to quickly create different kinds of wires which can change with the configured style.
{\catcode`\|=12 % Ensures | is not anymore \verb|...|
% Do not indent not to put space in final code
\begin{codeexample}[]
Spider \zx{\zxZ{\alpha}}, equation $\zx{\zxZ{}} = \zx{\zxX{}}$ %
and custom diagram: %
\begin{ZX}[red]
  \zxZ{\beta} \arrow[r]                           & \zxZ{\alpha} \\
  |[fill=pink,draw]| \gamma \arrow[ru,bend right]
\end{ZX}
\end{codeexample}
}
\end{pgfmanualentry}

\subsection{Nodes}

The following commands are useful to create different kinds of nodes. Always add empty arguments like |\example{}| if none are already present, otherwise if you type |\example| we don't guarantee backward compatibility.

\begin{command}{\zxEmptyDiagram{}}
  Create an empty diagram.
\begin{codeexample}[width=3cm]
\begin{ZX}
  \zxEmptyDiagram{}
\end{ZX}
\end{codeexample}
\end{command}


\begin{pgfmanualentry}
  \extractcommand\zxNone\opt{+}\opt{-}\marg{text}\@@
  \extractcommand\zxN\opt{+}\opt{-}\marg{text}\@@
  \pgfmanualbody
  Adds an empty node with |\zxNone{}| (alias |\zxN{}|). Combine it with wires and the |wire centered| style for straight lines or |C|-like wires (alias |wc|, or you will get holes) or |between none| style (alias |bn|) for other curved lines. Moreover, you should also add column and row spacing |&[\zxWCol]| and |\\[\zxWRow]| to avoid too shrinked diagrams when only wires are involved. The \verb#-|+# decorations are used to add a bit of (respectively) horizontal (\verb#\zxNone-{}#), vertical (\verb#\zxNone|{}#) and both (\verb#\zxNone+{}#) spacing (I don't know how to add \verb#|# in the documentation of the function).
\begin{codeexample}[width=3cm]
\begin{ZX}
  \zxNone{} \ar[C,d,wc] \ar[rd,s,bn] &[\zxWCol] \zxNone{}\\[\zxWRow]
  \zxNone{}             \ar[ru,s,bn] &          \zxNone{}
\end{ZX}
\end{codeexample}
\end{pgfmanualentry}


\begin{command}{\zxNoneDouble\opt{+-}\marg{text}}
  Like |\zxNone|, but the spacing for \verb#+-|# is large enough to fake two lines in only one. Not extremely useful (or one needs to play with |start anchor=south,end anchor=north|).
\begin{codeexample}[width=3cm]
\begin{ZX}
  \zxNoneDouble|{} \ar[r,s,start anchor=north,end anchor=south,ls=1.2] \ar[r,s,start anchor=south,end anchor=north,ls=1.2] &[\zxWCol] \zxNoneDouble|{}
\end{ZX}
\end{codeexample}
\end{command}

\begin{command}{\zxFracZ\opt{-}\marg{numerator}\opt{\oarg{numerator with parens}\oarg{denominator with parens}}\marg{denominator}}
  Adds a Z node with a fraction, use the minus decorator to add a small minus in front (the default minus is very big).
\begin{codeexample}[width=3cm]
\begin{ZX}
  \zxFracZ{\pi}{2} & \zxFracZ-{\pi}{2}
\end{ZX}
\end{codeexample}
The optional arguments are useful when the numerator or the denominator need parens when they are written inline (in that case optional arguments must be specified): it will prove useful when using a style that writes the fraction inline, for instance the default style for labels:
\begin{codeexample}[]
Compare
\begin{ZX}
  \zxFracZ{a+b}[(a+b)][(c+d)]{c+d}
\end{ZX} with %
\begin{ZX}[phase in label right]
  \zxFracZ{a+b}[(a+b)][(c+d)]{c+d}
\end{ZX}
\end{codeexample}
\end{command}

\begin{command}{\zxFracX\opt{-}\marg{numerator}\marg{denominator}}
  Adds an X node with a fraction.
\begin{codeexample}[width=3cm]
\begin{ZX}
  \zxFracX{\pi}{2} & \zxFracX-{\pi}{2}
\end{ZX}
\end{codeexample}
\end{command}


\begin{command}{\zxZ\opt{\oarg{other styles}}\marg{text}}
  Adds a Z node.
\begin{codeexample}[width=3cm]
\begin{ZX}
  \zxZ{} & \zxZ{\alpha} & \zxZ{\alpha + \beta} & \zxZ[text=red]{(a \oplus b)\pi}
\end{ZX}
\end{codeexample}
\end{command}

\begin{command}{\zxX\opt{\oarg{other styles}}\marg{text}}
  Adds an X node.
\begin{codeexample}[width=3cm]
\begin{ZX}
  \zxX{} & \zxX{\alpha} & \zxX{\alpha + \beta} & \zxX[text=green]{(a \oplus b)\pi}
\end{ZX}
\end{codeexample}
\end{command}

\begin{command}{\zxH\opt{\oarg{other styles}}}
  Adds an Hadamard node. See also |H| wire style.
\begin{codeexample}[width=3cm]
\begin{ZX}
  \zxNone{} \rar & \zxH{} \rar & \zxNone{}
\end{ZX}
\end{codeexample}
\end{command}



\begin{command}{\leftManyDots\opt{\oarg{text scale}\oarg{dots scale}}\marg{text}}
  Shortcut to add a dots and a text next to it. It automatically adds the new column, see more examples below.
\begin{codeexample}[]
\begin{ZX}
  \leftManyDots{n} \zxX{\alpha}
\end{ZX}
\end{codeexample}
\end{command}

\begin{command}{\leftManyDots\opt{\oarg{text scale}\oarg{dots scale}}\marg{text}}
  Shortcut to add a dots and a text next to it. It automatically adds the new column, see more examples below.
\begin{codeexample}[width=3cm]
\begin{ZX}
  \zxX{\alpha} \rightManyDots{m}
\end{ZX}
\end{codeexample}
\end{command}

\begin{command}{\middleManyDots{}}
  Shortcut to add a dots and a text next to it. It automatically adds the new column, see more examples below.
\begin{codeexample}[width=3cm]
\begin{ZX}
  \zxX{\alpha} \middleManyDots{} & \zxX{\beta}
\end{ZX}
\end{codeexample}
\end{command}

\begin{command}{\zxLoop\opt{\oarg{direction angle}\oarg{opening angle}\oarg{other styles}}}
  Adds a loop in \meta{direction angle} (defaults to $90$), with opening angle \meta{opening angle} (defaults to $20$).
\begin{codeexample}[width=3cm]
\begin{ZX}
  \zxX{\alpha} \zxLoop{} & \zxX{} \zxLoop[45]{} & \zxX{} \zxLoop[0][30][red]{}
\end{ZX}
\end{codeexample}
\end{command}

\begin{command}{\zxLoopAboveDots\opt{\oarg{opening angle}\oarg{other styles}}}
  Adds a loop above the node with some dots.
\begin{codeexample}[width=3cm]
\begin{ZX}
  \zxX{\alpha} \zxLoopAboveDots{}
\end{ZX}
\end{codeexample}
\end{command}

\noindent The previous commands can be useful to create this figure:
\begin{codeexample}[width=0pt]% Forces code/example on two lines.
\begin{ZX}
  \leftManyDots{n} \zxX{\alpha} \zxLoopAboveDots{} \middleManyDots{} \ar[r,o'=60]
      & \zxX{\beta} \zxLoopAboveDots{} \rightManyDots{m}
\end{ZX}
\end{codeexample}

\subsection{Phase in label style}

We also provide styles to place the phase on a label next to an empty node (not yet very well tested):

\begin{pgfmanualentry}
  \makeatletter
  \def\extrakeytext{style, }
  \extractkey/zx/styles rounded style/phase in content\@nil%
  \extractkey/zx/styles rounded style/phase in label=style (default {})\@nil%
  \extractkey/zx/styles rounded style/pil=style (default {})\@nil%
  \extractkey/zx/styles rounded style/phase in label above=style (default {})\@nil%
  \extractkey/zx/styles rounded style/pila=style (default {})\@nil%
  \extractkey/zx/styles rounded style/phase in label below=style (default {})\@nil%
  \extractkey/zx/styles rounded style/pilb=style (default {})\@nil%
  \extractkey/zx/styles rounded style/phase in label right=style (default {})\@nil%
  \extractkey/zx/styles rounded style/pilr=style (default {})\@nil%
  \extractkey/zx/styles rounded style/phase in label left=style (default {})\@nil%
  \extractkey/zx/styles rounded style/pill=style (default {})\@nil%
  \makeatother
  \pgfmanualbody
  The above styles are useful to place a spider phase in a label outside the node. They can either be put on the style of a node to modify a single node at a time:
\begin{codeexample}[]
  \zx{\zxX[phase in label]{\alpha} \rar & \zxX{\alpha}}
\end{codeexample}
\noindent It can also be configured on a per-figure basis:
\begin{codeexample}[]
\zx[phase in label right]{
  \zxZ{\alpha} \dar \\
  \zxX{\alpha} \dar \\
  \zxZ{}}
\end{codeexample}
\noindent or globally:
\begin{codeexample}[]
{
  \tikzset{
    /zx/user post preparation labels/.style={
      phase in label={label position=-45, text=purple,fill=none}
    }
  }
  \zx{
    \zxFracX-{\pi}{2}
  }
}
\end{codeexample}
Note that we must use |user post preparation labels| and not |/zx/user overlay nodes| because this will be run after all the machinery for labels has been setup.

  While |phase in content| forces the content of the node to be inside the node instead of inside a label (which is the default behavior), all other styles are special cases of |phase in label|. The \meta{style} parameter can be any style made for a tikz label:
\begin{codeexample}[width=3cm]
  \zx{
    \zxX[phase in label={label position=45, text=purple}]{\alpha}
  }
\end{codeexample}

For ease of use, the special cases of label position |above|, |below|, |right| and |left| have their respective shortcut style. The |pil*| versions are shortcuts of the longer style written above. For instance, |pilb| stands for |phase in label below|. Note also that by default labels will take some space, but it's possible to make them overlay without taking space using the |overlay| label style\dots{} however do it at your own risks as it can overlay the content around (also the text before and after):
\begin{codeexample}[width=0pt]
  \zx{
    \zxZ[pilb]{\alpha+\beta} \rar & \zxX[pilb]{\gamma} \rar & \zxZ[pilb=overlay]{\gamma+\eta}
  }
\end{codeexample}
The above also works for fractions:
\begin{codeexample}[]
\zx{\zxFracX[pilr]-{\pi}{2}}
\end{codeexample}
For fractions, you can configure how you want the label text to be displayed, either in a single line (default) or on two lines, like in nodes. The function |\zxConvertToFracInLabel| is in charge of that conversion, and can be changed to your needs to change this option document-wise. To use the same notation in both content and labels, you can do:
\begin{codeexample}[width=0pt]
  Compare
  \begin{ZX}[phase in label right]
    \zxFracZ{\pi}{2} \dar \\
    \zxFracZ{a+b}[(a+b)][(c+d)]{c+d}
  \end{ZX} with
{\RenewExpandableDocumentCommand{\zxConvertToFracInLabel}{mmmmm}{
    \zxConvertToFracInContent{#1}{#2}{#3}{#4}{#5}%
  }
  \begin{ZX}[phase in label right]
    \zxFracZ{\pi}{2} \dar \\
    \zxFracZ{a+b}[(a+b)][(c+d)]{c+d}
  \end{ZX} (exact same code!)
}
\end{codeexample}
Note that in |\zxFracZ{a+b}[(a+b)][(c+d)]{c+d}| the optional arguments are useful to put parens appropriately when the fraction is written inline.
\end{pgfmanualentry}


\subsection{Wires}

\begin{pgfmanualentry}
  \extractcommand\arrow\opt{\oarg{options}}\@@
  \extractcommand\ar\opt{\oarg{options}}\@@
  \pgfmanualbody
  These synonym commands (actually coming from |tikz-cd|) are used to draw wires between nodes. We refer to |tikz-cd| for an in-depth documentation, but what is important for our purpose is that the direction of the wires can be specified in the \meta{options} using a string of letters |r| (right), |l| (left), |u| (up), |d| (down).
\begin{codeexample}[]
\zx{\zxZ{} \ar[r] & \zxX{}} = \zx{\zxX{} \arrow[rd] \\ & \zxZ{}}
\end{codeexample}
  \meta{options} can also be used to add any additional style, either custom ones, or the ones defined in this library (this is recommended since it can be easily changed document-wise by simply changing the style). Multiple wires can be added in the same cell. Other shortcuts provided in |tikz-cd| like |\rar|\dots{} can be used.
{\catcode`\|=12 % Ensures | is not anymore \verb|...|
\begin{codeexample}[]
\begin{ZX}
  \zxZ{\alpha} \arrow[d, C] % C = Bell-like wire
               \ar[r,H,o']  % o' = top part of circle
               % H adds Hadamard, combine with \zxHCol
               \ar[r,H,o.] &[\zxHCol] \zxZ{\gamma}\\
  \zxZ{\beta}  \rar        & \zxX{} \ar[ld,red,"\circ" {marking,blue}] \ar[rd,s] \\
  \zxFracX-{\pi}{4}        & &\zxZ{}
\end{ZX}
\end{codeexample}
}
\end{pgfmanualentry}

As explained in |tikz-cd|, there are further shortened forms:
\begin{pgfmanualentry}
  \extractcommand\rar\opt{\oarg{options}}\@@
  \extractcommand\lar\opt{\oarg{options}}\@@
  \extractcommand\dar\opt{\oarg{options}}\@@
  \extractcommand\uar\opt{\oarg{options}}\@@
  \extractcommand\drar\opt{\oarg{options}}\@@
  \extractcommand\urar\opt{\oarg{options}}\@@
  \extractcommand\dlar\opt{\oarg{options}}\@@
  \extractcommand\ular\opt{\oarg{options}}\@@
  \pgfmanualbody
\end{pgfmanualentry}
The first one is equivalent to
\begin{verse}
  |\arrow|{\oarg{options}}|{r}|
\end{verse}
and the other ones work analogously.

We give now a list of wire styles provided in this library (|/zx/wires definition/| is an automatically loaded style). We recommend using them instead of manual styling to ensure they are the same document-wise, but they can of course be customized to your need. Note that the name of the styles are supposed to graphically represent the action of the style, and some characters are added to precise the shape: typically |'| means top, |.| bottom, |X-| is right to X (or should arrive with angle 0), |-X| is left to X (or should leave with angle zero). These shapes are usually designed to work when the starting node is left most (or above of both nodes have the same column). But they may work both way for some of them.


\begin{pgfmanualentry}
  \makeatletter
  \def\extrakeytext{style, }
  \extractkey/zx/wires definition/C\@nil%
  \extractkey/zx/wires definition/C.\@nil%
  \extractkey/zx/wires definition/C'\@nil%
  \extractkey/zx/wires definition/C-\@nil%
  \makeatother
  \pgfmanualbody
  Bell-like wires with an arrival at ``right angle'', |C| represents the shape of the wire, while |.| (bottom), |'| (top) and |-| (side) represent (visually) its position. Combine with |wire centered| (|wc|) to avoid holes when connecting multiple wires.
\begin{codeexample}[]
  Bell pair \zx{\zxNone{} \ar[d,C,wc] \\[\zxWRow]
                \zxNone{}}
  and funny graph
  \begin{ZX}
    \zxX{} \ar[d,C] \ar[r,C']  & \zxZ{} \ar[d,C-]\\
    \zxZ{} \ar[r,C.]           & \zxX{}
  \end{ZX}.
\end{codeexample}
\end{pgfmanualentry}

\begin{pgfmanualentry}
  \makeatletter
  \def\extrakeytext{style, }
  \extractkey/zx/wires definition/o'=angle (default 40)\@nil%
  \extractkey/zx/wires definition/o.=angle (default 40)\@nil%
  \extractkey/zx/wires definition/o-=angle (default 40)\@nil%
  \extractkey/zx/wires definition/-o=angle (default 40)\@nil%
  \makeatother
  \pgfmanualbody
  Curved wire, similar to |C| but with a soften angle (optionally specified via \meta{angle}, and globally editable with |\zxDefaultLineWidth|). Again, the symbols specify which part of the circle (represented with |o|) must be kept.
\begin{codeexample}[width=3cm]
  \begin{ZX}
    \zxX{} \ar[d,-o] \ar[d,o-]\\
    \zxZ{} \ar[r,o'] \ar[r,o.] & \zxX{}
  \end{ZX}.
\end{codeexample}
 Note that these wires can be combined with |H|, |X| or |Z|, in that case one should use appropriate column and row spacing as explained in their documentation:
\begin{codeexample}[width=3cm]
  \begin{ZX}
    \zxX{\alpha} \ar[d,-o,H] \ar[d,o-,H]\\[\zxHRow]
    \zxZ{\beta} \rar & \zxZ{} \ar[r,o',X] \ar[r,o.,Z] &[\zxSCol] \zxX{}
  \end{ZX}.
\end{codeexample}
\end{pgfmanualentry}

\begin{pgfmanualentry}
  \makeatletter
  \def\extrakeytext{style, }
  \extractkey/zx/wires definition/(\@nil%
  \extractkey/zx/wires definition/)\@nil%
  \makeatother
  \pgfmanualbody
  Curved wire, similar to |o| but can be used for diagonal items. The |(| and |)| symbols must be imagined as if the starting point was on top of the parens, and the ending point at the bottom.
\begin{codeexample}[width=3cm]
  \begin{ZX}
    \zxX{} \ar[rd,(] \ar[rd,),red]\\
    & \zxZ{}
  \end{ZX}.
\end{codeexample}
\end{pgfmanualentry}

\begin{pgfmanualentry}
  \makeatletter
  \def\extrakeytext{style, }
  \extractkey/zx/wires definition/s\@nil%
  \extractkey/zx/wires definition/s'=angle (default 30)\@nil%
  \extractkey/zx/wires definition/s.=angle (default 30)\@nil%
  \extractkey/zx/wires definition/-s'=angle (default 30)\@nil%
  \extractkey/zx/wires definition/-s.=angle (default 30)\@nil%
  \extractkey/zx/wires definition/s'-=angle (default 30)\@nil%
  \extractkey/zx/wires definition/s.-=angle (default 30)\@nil%
  \makeatother
  \pgfmanualbody
  |s| is used to create a s-like wire, to have nicer soften diagonal lines between nodes. Other versions are soften versions (the input and output angles are not as sharp, and the difference angle can be configured as an argument or globally using |\zxDefaultSoftAngleS|). Adding |'| or |.| specifies if the wire is going up-right or down-right.
\begin{codeexample}[width=3cm]
  \begin{ZX}
    \zxX{\alpha} \ar[s,rd] \\
                           & \zxZ{\beta}\\
    \zxX{\alpha} \ar[s.,rd] \\
                           & \zxZ{\beta}\\
                           & \zxZ{\alpha}\\
    \zxX{\beta} \ar[s,ru] \\
                           & \zxZ{\alpha}\\
    \zxX{\beta} \ar[s',ru] \\
  \end{ZX}
\end{codeexample}
|-| forces the angle on the side of |-| to be horizontal.
\begin{codeexample}[width=3cm]
  \begin{ZX}
    \zxX{\alpha} \ar[s.,rd] \\
                           & \zxZ{\beta}\\
    \zxX{\alpha} \ar[-s.,rd] \\
                           & \zxZ{\beta}\\
    \zxX{\alpha} \ar[s.-,rd] \\
                           & \zxZ{\beta}\\
  \end{ZX}
\end{codeexample}
\end{pgfmanualentry}

\begin{pgfmanualentry}
  \makeatletter
  \def\extrakeytext{style, }
  \extractkey/zx/wires definition/ss\@nil%
  \extractkey/zx/wires definition/ss.=angle (default 30)\@nil%
  \extractkey/zx/wires definition/.ss=angle (default 30)\@nil%
  \extractkey/zx/wires definition/sIs.=angle (default 30)\@nil%
  \extractkey/zx/wires definition/.sIs=angle (default 30)\@nil%
  \extractkey/zx/wires definition/ss.I-=angle (default 30)\@nil%
  \extractkey/zx/wires definition/I.ss-=angle (default 30)\@nil%
  \makeatother
  \pgfmanualbody
  |ss| is similar to |s| except that we go from top to bottom instead of from left to right. The position of |.| says if the node is wire is going bottom right (|ss.|) or bottom left (|.ss|).
\begin{codeexample}[width=3cm]
  \begin{ZX}
    \zxX{\alpha} \ar[ss,rd] \\
                           & \zxZ{\beta}\\
    \zxX{\alpha} \ar[ss.,rd] \\
                           & \zxZ{\beta}\\
                           & \zxX{\beta} \ar[.ss,dl] \\
    \zxZ{\alpha}\\
                           & \zxX{\beta} \ar[.ss=40,dl] \\
    \zxZ{\alpha}\\
  \end{ZX}
\end{codeexample}
|I| forces the angle above (if in between the two |s|) or below (if on the same side as |.|) to be vertical.
\begin{codeexample}[width=3cm]
  \begin{ZX}
    \zxX{\alpha} \ar[ss,rd] \\
                           & \zxZ{\beta}\\
    \zxX{\alpha} \ar[sIs.,rd] \\
                           & \zxZ{\beta}\\
    \zxX{\alpha} \ar[ss.I,rd] \\
                           & \zxZ{\beta}\\
                           & \zxX{\beta} \ar[.sIs,dl] \\
    \zxZ{\alpha}\\
                           & \zxX{\beta} \ar[I.ss,dl] \\
    \zxZ{\alpha}\\
  \end{ZX}
\end{codeexample}
\end{pgfmanualentry}

\begin{pgfmanualentry}
  \makeatletter
  \def\extrakeytext{style, }
  \extractkey/zx/wires definition/3 dots=text (default =)\@nil%
  \extractkey/zx/wires definition/3 vdots=text (default =)\@nil%
  \makeatother
  \pgfmanualbody
  The styles put in the middle of the wire (without drawing the wire) $\dots$ (for |3 dots|) or $\vdots$ (for |3 vdots|). The dots are scaled according to |\zxScaleDots| and the text \meta{text} is written on the left. Use |&[\zxDotsRow]| and |\\[\zxDotsRow]| to properly adapt the spacing of columns and rows.
\begin{codeexample}[width=3cm]
\begin{ZX}
  \zxZ{\alpha} \ar[r,o'] \ar[r,o.]
               \ar[r,3 dots]
               \ar[d,3 vdots={$n$\,}] &[\zxDotsCol] \zxFracX{\pi}{2}\\[\zxDotsRow]
  \zxZ{\alpha} \rar             & \zxFracX{\pi}{2}
\end{ZX}
\end{codeexample}
\end{pgfmanualentry}

\begin{pgfmanualentry}
  \makeatletter
  \def\extrakeytext{style, }
  \extractkey/zx/wires definition/H\@nil%
  \extractkey/zx/wires definition/Z\@nil%
  \extractkey/zx/wires definition/X\@nil%
  \makeatother
  \pgfmanualbody
  Adds a |H| (Hadamard), |Z| or |X| node (without phase) in the middle of the wire. Width of column or rows should be adapted accordingly using |\zxNameRowcolFlatnot| where |Name| is replaced by |H|, |S| (for ``spiders'', i.e.\ |X| or |Z|), |HS| (for both |H| and |S|) or |W|, |Rowcol| is either |Row| (for changing row sep) or |Col| (for changing column sep) and |Flatnot| is empty or |Flat| (if the wire is supposed to be a straight line as it requires more space). For instance:
\begin{codeexample}[width=3cm]
\begin{ZX}
  \zxZ{\alpha} \ar[d] \ar[r,o',H] \ar[r,o.,H] &[\zxHCol] \zxX{\beta}\\
  \zxZ{\alpha}  \ar[d,-o,X] \ar[d,o-,Z]                        \\[\zxHSRow]
  \zxX{\gamma}
\end{ZX}
\end{codeexample}
\end{pgfmanualentry}

\begin{pgfmanualentry}
  \makeatletter
  \def\extrakeytext{style, }
  \extractkey/zx/wires definition/wire centered\@nil%
  \extractkey/zx/wires definition/wc\@nil%
  \extractkey/zx/wires definition/wire centered start\@nil%
  \extractkey/zx/wires definition/wcs\@nil%
  \extractkey/zx/wires definition/wire centered end\@nil%
  \extractkey/zx/wires definition/wce\@nil%
  \makeatother
  \pgfmanualbody
  When the wires are drawn, they start from the border of the node. However, it can make strange results if nodes do not have the same size, or if we connect none nodes (we will get holes). It is therefore possible to force the wire to start at the center of the node (|wire centered start|, alias |wcs|), to end at the center of the node -|wire centered end|, alias |wce|) or both (|wire centered|, alias |wc|). See also |between node| to also increase looseness when connecting only wires.
\begin{codeexample}[width=3cm]
\begin{ZX}
  \zxZ{} \ar[o',r] \ar[o.,r]       & \zxX{\alpha}\\
  \zxZ{} \ar[o',r,wc] \ar[o.,r,wc] & \zxX{\alpha}
\end{ZX}
\end{codeexample}
Without |wc| (note that because there is no node, we need to use |&[\zxWCol]| (for columns) and |\\[\zxWRow]| (for rows) to get nicer spacing):
\begin{codeexample}[width=3cm]
\zx{\zxNone{} \rar &[\zxWCol] \zxNone{} \rar &[\zxWCol] \zxNone{} }
\end{codeexample}
With |wc|:
\begin{codeexample}[width=3cm]
\zx{\zxNone{} \rar[wc] &[\zxWCol] \zxNone{} \rar[wc] &[\zxWCol] \zxNone{}}
\end{codeexample}
\end{pgfmanualentry}

\begin{stylekey}{/zx/wires definition/ls=looseness}
  Shortcut for |looseness|, allows to quickly redefine looseness. Use with care (or redefine style directly). Note that you can also change yourself other values, like |in|, |out|\dots
\begin{codeexample}[]
\begin{ZX}
  \zxZ{} \ar[rd,s] \\
                   & \zxX{}\\
  \zxZ{} \ar[rd,s,ls=3] \\
                   & \zxX{}
\end{ZX}
\end{codeexample}
\end{stylekey}

\begin{pgfmanualentry}
  \makeatletter
  \def\extrakeytext{style, }
  \extractkey/zx/wires definition/between none\@nil%
  \extractkey/zx/wires definition/bn\@nil%
  \makeatother
  \pgfmanualbody
  When drawing wires only, the default looseness may not be good looking and holes may appear in the line. This style (whose alias is |bn|) should therefore be used when curved wires (except |C| which already has a good looseness, use |wire centered| instead) are connected together. In that case, also make sure to separate columns using |&[\zxWCol]| and row using |\\[\zxWRow]|.
\begin{codeexample}[width=3cm]
A swapped Bell pair is %
\begin{ZX}
  \zxNone{} \ar[C,d,wc] \ar[rd,s,bn] &[\zxWCol] \zxNone{} \\[\zxWRow]
  \zxNone{}             \ar[ru,s,bn] &          \zxNone{}
\end{ZX}
\end{codeexample}
\end{pgfmanualentry}

\section{Advanced styling}

\subsection{Overlaying or creating styles}

It is possible to arbitrarily customize the styling, create or update ZX or tikz styles\dots{} First, any option that can be given to a |tikz-cd| matrix can also be given to a |ZX| environment (we refer to the manual of |tikz-cd| for more details). We also provide overlays to quickly modify the ZX style.

\begin{stylekey}{/zx/default style nodes}
  This is where the default style must be loaded. By default, it simply loads the (nested) style packed with this library, |/zx/styles/rounded style|. You can change the style here if you would like to globally change a style.

  Note that a style must typically define at least |zxZ2|, |zxX2|, |zxFracZ6|, |zxFracX6|, |\zxH|, |zxHSmall|, |zxNoPhaseSmallZ|, |zxNoPhaseSmallX|, |zxNone{,+,-,I}|, |zxNoneDouble{,+,-,I}| and all the |phase in label*|, |pil*| styles (see code on how to define them). Because the above styles (notably |zxZ*| and |zxFrac*|) are slightly complex to define (this is needed in order to implement |phase in label|), it may be quite long to implement them all properly by yourself.

  For that reason, it may be easier to load our default style and overlay only some of the styles we use (see example in |/zx/user overlay nodes| right after). You can check our code in |/zx/styles/rounded style| to see what you can redefine (intuitively, the styles like |my style name| should be callable by the end user, |myStyleName| may be redefined by users or used in tikzit, and |my@style@name| are styles that should not be touched by the user). The styles that have most interests are |zxNoPhase| (for Z and X nodes without any phase inside), |zxShort| (for Z and X nodes for fractions typically), |zxLong| (for other Z and X nodes) and |stylePhaseInLabel| (for labels when using |phase in label|). These basic styles are extended to add colors (just add |Z|/|X| after the name) like |zxNoPhaseZ|\dots{} You can change them, but if you just want to change the color, prefer to redefine |colorZxZ|/|colorZxZ| instead (note that this color does not change |stylePhaseInLabelZ/X|, so you are free to redefine these styles as well). All the above styles can however be called from inside a tikzit style, if you want to use tikzit internally (make sure to load this library then in |*.tikzdefs|).

  Note however that you should avoid to call these styles from inside |\zx{...}| since |\zx*| and |\zxFrac*| are supposed to choose automatically the good style for you depending on the mode (fractions, labels in phase\dots{}). For more details, we encourage the advanced users too look at the code of the library, and examples for simple changes will be presented now.
\end{stylekey}

\begin{stylekey}{/zx/user overlay nodes}
  If a user just wants to overlay some parts of the node styles, add your changes here.
\begin{codeexample}[]
  {\tikzset{
      /zx/user overlay nodes/.style={
        zxH/.append style={dashed,inner sep=2mm}
      }}
    \zx{\zxNone{} \rar & \zxH{} \rar & \zxNone{}}
  }
\end{codeexample}
You can also change it on a per-diagram basis:
\begin{codeexample}[]
  \zx[text=yellow,/zx/user overlay nodes/.style={
    zxSpiders/.append style={thick,draw=purple}}
  ]{\zxX{} \rar & \zxX{\alpha} \rar & \zxFracZ-{\pi}{2}}
\end{codeexample}
The list of keys that can be changed will be given below in |/zx/styles/rounded style/*|.
\end{stylekey}

\begin{stylekey}{/zx/default style wires}
  Default style for wires. Note that |/zx/wires definition/| is always loaded by default, and we don't add any other style for wires by default. But additional styles may use this functionality.
\end{stylekey}

\begin{stylekey}{/zx/user overlay wires}
  The user can add here additional styles for wires.
\begin{codeexample}[]
\begin{ZX}[/zx/user overlay wires/.style={thick,->,C/.append style={dashed}}]
  \zxNone{} \ar[d,C] \rar[] &[\zxWCol] \zxNone{}\\[\zxWRow]
  \zxNone{} \rar[] & \zxNone{}
\end{ZX}
\end{codeexample}
\end{stylekey}

\begin{stylekey}{/zx/styles/rounded style}
  This is the style loaded by default. It contains internally other (nested) styles that must be defined for any custom style.
\end{stylekey}

We present now all the properties that a new node style must have (and that can overlayed as explained above).
\begin{stylekey}{/zx/styles/rounded style/zxAllNodes}
  Style applied to all nodes.
\end{stylekey}

\begin{stylekey}{/zx/styles/rounded style/zxEmptyDiagram}
  Style to draw an empty diagram.
\end{stylekey}

\begin{pgfmanualentry}
  \makeatletter
  \def\extrakeytext{style, }
  \extractkey/zx/styles/rounded style/zxNone\@nil%
  \extractkey/zx/styles/rounded style/zxNone+\@nil%
  \extractkey/zx/styles/rounded style/zxNone-\@nil%
  \extractkey/zx/styles/rounded style/zxNoneI\@nil%
  \makeatother
  \pgfmanualbody
  Styles for None wires (no inner sep, useful to connect to wires). The |-|,|I|,|+| have additional horizontal, vertical, both spaces.
\end{pgfmanualentry}

\begin{pgfmanualentry}
  \makeatletter
  \def\extrakeytext{style, }
  \extractkey/zx/styles/rounded style/zxNoneDouble\@nil%
  \extractkey/zx/styles/rounded style/zxNoneDouble+\@nil%
  \extractkey/zx/styles/rounded style/zxNoneDouble-\@nil%
  \extractkey/zx/styles/rounded style/zxNoneDoubleI\@nil%
  \makeatother
  \pgfmanualbody
  Like |zxNone|, but with more space to fake two nodes on a single line (not very used).
\end{pgfmanualentry}

\begin{stylekey}{/zx/styles/rounded style/zxSpiders}
  Style that apply to all circle spiders.
\end{stylekey}

\begin{stylekey}{/zx/styles/rounded style/zxNoPhase}
  Style that apply to spiders without any angle inside. Used by |\zxX{}| when the argument is empty.
\end{stylekey}

\begin{stylekey}{/zx/styles/rounded style/zxNoPhaseSmall}
  Like |zxNoPhase| but for spiders drawn in between wires.
\end{stylekey}

\begin{stylekey}{/zx/styles/rounded style/zxShort}
  Spider with text but no inner space. Used notably to obtain nice fractions.
\end{stylekey}

\begin{stylekey}{/zx/styles/rounded style/zxLong}
  Spider with potentially large text. Used by |\zxX{\alpha}| when the argument is not empty.
\end{stylekey}

\begin{pgfmanualentry}
  \makeatletter
  \def\extrakeytext{style, }
  \extractkey/zx/styles/rounded style/zxNoPhaseZ\@nil%
  \extractkey/zx/styles/rounded style/zxNoPhaseX\@nil%
  \extractkey/zx/styles/rounded style/zxNoPhaseSmallZ\@nil%
  \extractkey/zx/styles/rounded style/zxNoPhaseSmallX\@nil%
  \extractkey/zx/styles/rounded style/zxShortZ\@nil%
  \extractkey/zx/styles/rounded style/zxShortX\@nil%
  \extractkey/zx/styles/rounded style/zxLongZ\@nil%
  \extractkey/zx/styles/rounded style/zxLongX\@nil%
  \makeatother
  \pgfmanualbody
  Like above styles, but with colors of |X| and |Z| spider added. The color can be changed globally by updating the |colorZxX| color. By default we use:
  \begin{verse}
    |\definecolor{colorZxZ}{RGB}{204,255,204}|\\
    |\definecolor{colorZxX}{RGB}{255,136,136}|\\
    |\definecolor{colorZxH}{RGB}{255,255,0}|
  \end{verse}
  as the second recommendation in \href{https://zxcalculus.com/accessibility.html}{\texttt{zxcalculus.com/accessibility.html}}.
\end{pgfmanualentry}

\begin{stylekey}{/zx/styles/rounded style/zxH}
  Style for Hadamard spiders, used by |\zxH{}| and uses the color |colorZxH|.
\end{stylekey}

\begin{stylekey}{/zx/styles/rounded style/zxHSmall}
  Like |zxH| but for Hadamard on wires, (see |H| style).
\end{stylekey}

\begin{pgfmanualentry}
  \extractcommand\zxConvertToFracInContent\marg{sign}\marg{num no parens}\marg{denom no parens}\marg{nom parens}\marg{denom parens}\@@
  \extractcommand\zxConvertToFracInLabel\@@
  \pgfmanualbody
  These functions are not meant to be used, but redefined using something like (we use |\zxMinus| as a shorter minus compared to $-$):
\begin{verse}
  |\RenewExpandableDocumentCommand{\zxConvertToFracInLabel}{mmmmm}{%|\\
  |  \ifthenelse{\equal{#1}{-}}{\zxMinus}{#1}\frac{#2}{#3}%|\\
  |}|
\end{verse}
This is used to change how the library does the conversion between |\zxFrac| and the actual written text (either in the node content or in the label depending on the function). The first argument is the sign (string |-| for minus, anything else must be written in place of the minus), the second and third argument are the numerator and denominator of the fraction when used in |\frac{}{}| while the last two arguments are the same except that they include the parens which should be added when using an inline version. For instance, one could get a call |\zxConvertToFracInLabel{-}{a+b}{c+d}{(a+b)}{(c+d)}|. See part on labels to see an example of use.
\end{pgfmanualentry}

\noindent We also define several spacing commands that can be redefined to your needs:
\begin{pgfmanualentry}
  \extractcommand\zxHCol{}\@@
  \extractcommand\zxHRow{}\@@
  \extractcommand\zxHColFlat{}\@@
  \extractcommand\zxHRowFlat{}\@@
  \extractcommand\zxSCol{}\@@
  \extractcommand\zxSRow{}\@@
  \extractcommand\zxSColFlat{}\@@
  \extractcommand\zxSRowFlat{}\@@
  \extractcommand\zxHSCol{}\@@
  \extractcommand\zxHSRow{}\@@
  \extractcommand\zxHSColFlat{}\@@
  \extractcommand\zxHSRowFlat{}\@@
  \extractcommand\zxWCol{}\@@
  \extractcommand\zxWRow{}\@@
  \extractcommand\zxDotsCol{}\@@
  \extractcommand\zxDotsRow{}\@@
  \pgfmanualbody
  These are spaces, to use like |&[\zxHCol]| or |\\[\zxHRow]| in order to increase the default spacing of rows and columns depending on the style of the wire. |H| stands for Hadamard, |S| for Spiders, |W| for Wires only, |HS| for both Spiders and Hadamard, |Dots| for the 3 dots styles. And of course |Col| for columns, |Row| for rows.
\end{pgfmanualentry}


\begin{pgfmanualentry}
  \extractcommand\zxDefaultSoftAngleS{}\@@
  \extractcommand\zxDefaultSoftAngleO{}\@@
  \pgfmanualbody
  Default opening angles of |S| and |o| wires. Defaults to respectively $30$ and $40$.
\end{pgfmanualentry}

\begin{command}{\zxMinus{}}
  The minus sign used in fractions.
\end{command}

\subsection{Further customization}

You can further customize your drawings using any functionality from \tikzname{} and |tikz-cd| (but it is of course at your own risk). For instance, we can define our own style to create blocks:
{\catcode`\|=12 % Ensures | is not anymore \verb|...|
\begin{codeexample}[width=0pt]
{ % \usetikzlibrary{shadows}
  \tikzset{
    my bloc/.style={
      anchor=center,
      inner sep=2pt,
      inner xsep=.7em,
      minimum height=3em,
      draw,
      thick,
      fill=blue!10!white,
      double copy shadow={opacity=.5},tape,
    }
  }
  \zx{|[my bloc]| f \rar &[1mm] |[my bloc]| g \rar &[1mm] \zxZ{\alpha} \rar & \zxNone{}}
}
\end{codeexample}
}
Or we can use for instance |fit|, |alias|, |execute at end picture| and layers (the user can use |background| for things behind the drawing, |box| for special nodes above the drawings (like multi-column nodes, see below), and |foreground| which is even higher) to do something like that:
{\catcode`\|=12 % Ensures | is not anymore \verb|...|
\begin{codeexample}[width=3cm]
% \usetikzlibrary{fit}
\begin{ZX}[
  execute at end picture={
    \node[
      inner sep=2pt,
      node on layer=background, %% Ensure the node is behind the drawing
      rounded corners,
      draw=blue,
      dashed,
      fill=blue!50!white,
      opacity=.5,
      fit=(cnot1)(cnot2), %% position of the node, thanks fit library
      "\textsc{cnot}" above %% Adds label, thanks quote library
    ] {};
  }
  ]
  \zxNone{} \rar & \zxZ[alias=cnot1]{} \dar \rar & \zxNone{}\\
  \zxNone{} \rar & \zxX[alias=cnot2]{} \rar      & \zxNone{}\\
\end{ZX}
\end{codeexample}
}
This can also be used to fake multi-columns nodes (I need to check later if I can facilitate this kind of operation from the library directly):
{\catcode`\|=12 % Ensures | is not anymore \verb|...|
\begin{codeexample}[width=3cm]
% \usetikzlibrary{fit}
\tikzset{
  my box/.style={inner sep=4pt, draw, thick, fill=white,anchor=center},
}
\begin{ZX}[
  execute at end picture={
    \node[
      my box,
      node on layer=box, %% Ensure the node is above the wires
      fit=(f1)(f2), %% position of the node, thanks fit library
      label={[node on layer=box]center:$f$},
    ] {};
  }
  ]
  \zxNoneDouble+[alias=f1]{} \rar &[1mm] |[my box]| g \rar & \zxNone{}\\
  \zxNoneDouble+[alias=f2]{} \rar &[1mm] \zxZ{} \rar       & \zxNone{}\\
\end{ZX}
\end{codeexample}
}
\section{Acknowledgement}

I'm very grateful of course to everybody that worked on these amazing field which made diagramatic quantum computing possible, and to the many StackExchange users (sorry, I don't want to risk an incomplete list) that helped me to understand a bit more \LaTeX and \tikzname. I also thank Robert Booth for making me realize how my old style was bad, and for giving advices on how to improve it. Thanks to John van de Wetering, whose style has also been a source of inspiration.

\printindex

\end{document}
